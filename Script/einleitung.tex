\chapter{Einleitung}
Der vorliegende Text ist ein Fragment eines Vorlesungs-Skriptes f�r die Analysis-Vorlesung f�r
Informatiker.  Ich habe mich bei der Ausarbeitung dieser Vorlesung im wesentlichen auf die folgenden
Quellen gest�tzt:
\begin{enumerate}
\item \emph{Analysis} \texttt{I} von Otto Forster \cite{forster:2011}.
\item \emph{Differential- und Integralrechnung \texttt{I}} von Hans Grauert und Ingo Lieb \cite{grauert:1967}.
\item \emph{Differential and Integral Calculus, Volume 1} von Richard Courant  \cite{courant:1937}.
\item \emph{Advanced Calculus} von Richard Wrede und Murray R.~Spiegel \cite{wrede:2010}.
\end{enumerate}
Den Studenten empfehle ich das erste Buch in dieser Liste, denn dieses Buch ist auch in
elektronischer Form in unserer Bibliothek vorhanden.  Bei dem Buch von Richard Courant ist das
Copyright mittlerweile abgelaufen, so dass Sie es im Netz unter
\\[0.2cm]
\hspace*{0.3cm}
\href{https://ia700700.us.archive.org/34/items/DifferentialIntegralCalculusVolI/Courant-DifferentialIntegralCalculusVolI.pdf}{\texttt{https://ia700700.us.archive.org/}
\\
\hspace*{0.3cm}
\texttt{34/items/DifferentialIntegralCalculusVolI/Courant-DifferentialIntegralCalculusVolI.pdf}}
\\[0.2cm]
finden k�nnen.  Schlie�lich enth�lt das Buch von Wrede und Spiegel eine Vielzahl gel�ster
Aufgaben und bietet sich daher besonders zum �ben an.

\section{�berblick �ber die Vorlesung}
Im Rahmen der Vorlesung werden die folgenden Gebiete behandelt:
\begin{enumerate}
\item Im zweiten Kapitel werden die reellen Zahlen mit Hilfe von Dedekind-Schnitten definiert.
\item Das dritte Kapitel f�hrt den Begriff des Grenzwerts f�r Folgen und Reihen ein.
\item Das vierte Kapitel diskutiert die Begriffe Stetigkeit und Differenzierbarkeit.
\item Das f�nfte Kapitel zeigt verschiedene Anwendungen der bis dahin dargestellten Theorie.
      Insbesondere werden \emph{Taylor-Reihen} diskutiert. Diese k�nnen beispielsweise zur
      Berechnung der trigonometrischen Funktionen verwendet werden.  Au�erdem diskutieren wir in
      diesem Kapitel Verfahren zur numerischen L�sung von Gleichungen.
\item Das sechste Kapitel besch�ftigt sich mit der Integralrechnung.
\item Im siebten Kapitel zeigen wir, dass  $\pi$ und $e$ keine rationalen Zahlen sind.
\item Im letzten Kapitel diskutieren wir Fourier-Reihen.
\end{enumerate}

\section{Ziel der Vorlesung}
Wir werden im Rahmen der Vorlesung nicht die Zeit haben, alle Aspekte der Analysis zu besprechen.
Insbesondere werden wir viele interessante Anwendungen der Analysis in der Informatik nicht
diskutieren k�nnen.  Das ist aber auch gar nicht das Ziel dieser Vorlesung:  Mir geht es vor allem
darum, Ihnen die F�higkeit zu vermitteln, sich 
selbstst�ndig in mathematische Fachliteratur hineinarbeiten zu k�nnen.  Dazu m�ssen Sie in der Lage
sein, mathematische Beweise sowohl zu verstehen als auch selber entwickeln zu k�nnen.   Dies ist ein
wesentlicher Unterschied zu der Mathematik, an die sich viele von Ihnen auf der Schule gew�hnthaben:
Dort werden prim�r Verfahren vermitteln, mit denen sich spezielle Probleme l�sen lassen.  Die Kenntnis
solcher Verfahren ist allerdings in der Praxis nicht mehr wichtig, denn heutzutage werden solche 
Verfahren programmiert und daher besteht kein Bedarf mehr daf�r, solche Verfahren von Hand
anzuwenden.  Aus diesem Grund wird in dieser Vorlesung der mathematische Beweis-Begriff im
Vordergrund stehen.  Die Analysis dient uns dabei als ein Beispiel einer mathematischen Theorie, an
Hand derer wir das mathematische Denken �ben k�nnen. 

\section{Notation}
In diesem Skript definieren wir die Menge der nat�rlichen Zahlen $\mathbb{N}$ �ber die Formel
\\[0.2cm]
\hspace*{1.3cm}
$\mathbb{N} := \{ 1, 2, 3, \cdots \}$.
\\[0.2cm]
Im Gegensatz zu der Vorlesung �ber Lineare Algebra im letzten Semester wird die Zahl $0$ in diesem
Skript also \underline{nicht} als nat�rliche Zahl aufgefasst.  Weiter definieren wir
\\[0.2cm]
\hspace*{1.3cm}
$\mathbb{N}_0 := \{ 0 \} \cup \mathbb{N}$.


\section{Eine Bitte}
Diese Skript enth�lt noch eine Menge Tipp-Fehler.  Sollte Ihnen ein Fehler auffallen, so bitte ich
um einen Hinweis unter der Adresse
\\[0.2cm]
\hspace*{1.3cm}
\href{mailto:karl.stroetmann@dhbw-mannheim.de}{karl.stroetmann@dhbw-mannheim.de}.
\\[0.2cm]
Wenn Sie mit \href{http://github.com}{\texttt{github}} vertraut sind, k�nnen Sie mir auch gerne einen
\href{https://help.github.com/articles/using-pull-requests}{\textsl{Pull Request}} schicken.


%%% Local Variables: 
%%% mode: latex
%%% TeX-master: "analysis"
%%% End: 
