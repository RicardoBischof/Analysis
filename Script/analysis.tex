\documentclass{report}
\usepackage{german}
\usepackage[latin1]{inputenc}
\usepackage{a4wide}
\usepackage{epsfig}
\usepackage{amssymb}
\usepackage{amsmath}
\usepackage{fancyvrb}
\usepackage{alltt}
\usepackage{fleqn}
\usepackage{epic}
\usepackage{color} 
\usepackage{theorem}
\usepackage{hyperref}
\usepackage[all]{hypcap}
\hypersetup{
        colorlinks = true, % comment this to make xdvi work
        linkcolor  = blue,
        citecolor  = red,
        filecolor  = [rgb]{0.1, 0.1, 1.0},
        urlcolor   = [rgb]{0.1, 0.1, 1.0},
        pdfborder  = {0 0 0} 
}

\usepackage{fancyhdr}
\usepackage{lastpage} 

\pagestyle{fancy}

\fancyfoot[C]{--- \thepage/\pageref{LastPage}\ ---}

\fancypagestyle{plain}{%
\fancyhf{}
\fancyfoot[C]{--- \thepage/\pageref{LastPage}\ ---}
\renewcommand{\headrulewidth}{0pt}
}

\renewcommand{\chaptermark}[1]{\markboth{\chaptername \ \thechapter.\ #1}{}}
\renewcommand{\sectionmark}[1]{\markright{\thesection. \ #1}{}}
\fancyhead[R]{\leftmark}
\fancyhead[L]{\rightmark}

\definecolor{amethyst}{rgb}{0.2, 0.4, 0.6}
\definecolor{orange}{rgb}{1, 0.9, 0.0}

\usepackage[german]{babel}
\usepackage{babelbib}
\bibliographystyle{babunsrt}

{\theorembodyfont{\sf}
\newtheorem{Definition}{Definition}
\newtheorem{Axiom}[Definition]{Axiom}
\newtheorem{Notation}[Definition]{Notation}
\newtheorem{Korollar}[Definition]{Korollar}
\newtheorem{Lemma}[Definition]{Lemma}
\newtheorem{Satz}[Definition]{Satz}
\newtheorem{Theorem}[Definition]{Theorem}
}

\newcommand{\proof}{\vspace*{0.2cm}

\noindent
\textbf{Beweis}: }
 
\newcommand{\qed}{\hspace*{\fill} $\Box$
\vspace*{0.2cm}

}

\newcommand{\eod}{\hspace*{\fill} $\diamond$
\vspace*{0.2cm}

}

\newcommand{\eox}{\hspace*{\fill} $\diamond$
\vspace*{0.2cm}

}

\title{Analysis\\[0.3cm]
      --- Sommersemester 2014 --- \\[0.3cm]
      DHBW Mannheim}
\author{Prof.~Dr.~Karl Stroetmann}
\date{\today \\[5.5cm]
\begin{minipage}[t]{1.0\linewidth}
\noindent
Dieses Skript ist einschlie�lich der \LaTeX-Quellen sowie der in diesem Skript diskutierten
Programme unter
\\[0.2cm]
\hspace*{\fill}
\href{https://github.com/karlstroetmann/Analysis}{\texttt{https://github.com/karlstroetmann/Analysis}}
\hspace*{\fill} 
\\[0.2cm]
im Netz verf�gbar.  Das Skript wird laufend �berarbeitet.  Wenn Sie auf Ihrem Rechner \href{http://git-scm.com/download}{\texttt{git}}
installieren und mein Repository mit Hilfe des Befehls
\\[0.2cm]
\hspace*{1.3cm}
\texttt{git clone https://github.com/karlstroetmann/Analysis.git}
\\[0.2cm]
klonen, dann k�nnen Sie durch Absetzen des Befehls
\\[0.2cm]
\hspace*{1.3cm}
\texttt{git pull}
\\[0.2cm]
die aktuelle Version meines Skripts aus dem Netz laden.
\end{minipage}
}


\newcommand{\solution}{\vspace*{0.2cm}

\noindent
\textbf{L�sung}: }

\newcounter{aufgabe}
\newcommand{\exercise}{\vspace*{0.2cm}
\stepcounter{aufgabe}

\noindent
\textbf{Aufgabe \arabic{aufgabe}}: }

\newcommand{\example}{\vspace*{0.2cm}

\noindent
\textbf{Beispiel}: \ }

\newcommand{\examples}{\vspace*{0.2cm}

\noindent
\textbf{Beispiele}: \ }
 
\newcommand{\remark}{\vspace*{0.2cm}
\noindent
\textbf{Bemerkung}: }

\newcommand{\lb}{\hspace*{\fill} \linebreak}

\newcommand{\bruch}[2]{\displaystyle\frac{\;\displaystyle#1\;}{\;\displaystyle#2\;}}
\newcommand{\bruchs}[2]{\textstyle\frac{\;\textstyle#1\;}{\;\textstyle#2\;}}
\newcommand{\folge}[1]{\bigl(#1\bigr)_{n\in\mathbb{N}}}
\newcommand{\folgea}[1]{\bigl(#1\bigr)_{n\in\mathbb{N}\backslash\{0\}}}
\newcommand{\Folge}[1]{\left(#1\right)_{n\in\mathbb{N}}}
\newcommand{\Reihe}[1]{\left(\sum\limits_{i=1}^n #1\right)_{n\in\mathbb{N}}}
\newcommand{\bint}{\displaystyle\int}
\newcommand{\dint}[2]{\displaystyle\int_{#1}^{#2}\hspace{-0.2cm}}
\newcommand{\Oh}{\mathcal{O}}
\newcommand{\df}{\displaystyle\frac{d\;}{dx}}
\newcommand{\err}[1]{\textsl{error}_n(#1)}
\newcommand{\erri}[2]{\textsl{error}^{(#2)}_n(#1)}
\newcommand{\norm}[1]{\big\|#1\bigr\|_{\infty}}

\def\pair(#1,#2){\langle #1, #2 \rangle}

\newlength{\mylength}
\setlength{\mathindent}{1.3cm}

%\includeonly{folgen-und-reihen}
%\includeonly{rundungsfehler}

\begin{document}

\maketitle
\tableofcontents
\chapter{Einleitung}
Der vorliegende Text ist ein Fragment eines Vorlesungs-Skriptes f�r die Analysis-Vorlesung f�r
Informatiker.  Ich habe mich bei der Ausarbeitung dieser Vorlesung im wesentlichen auf die folgenden
Quellen gest�tzt:
\begin{enumerate}
\item \emph{Analysis} \texttt{I} von Otto Forster \cite{forster:2011}.
\item \emph{Differential- und Integralrechnung \texttt{I}} von Hans Grauert und Ingo Lieb \cite{grauert:1967}.
\item \emph{Differential and Integral Calculus, Volume 1} von Richard Courant  \cite{courant:1937}.
\item \emph{Advanced Calculus} von Richard Wrede und Murray R.~Spiegel \cite{wrede:2010}.
\end{enumerate}
Den Studenten empfehle ich das erste Buch in dieser Liste, denn dieses Buch ist auch in
elektronischer Form in unserer Bibliothek vorhanden.  Bei dem Buch von Richard Courant ist das
Copyright mittlerweile abgelaufen, so dass Sie es im Netz unter
\\[0.2cm]
\hspace*{0.3cm}
\href{https://ia700700.us.archive.org/34/items/DifferentialIntegralCalculusVolI/Courant-DifferentialIntegralCalculusVolI.pdf}{\texttt{https://ia700700.us.archive.org/}
\\
\hspace*{0.3cm}
\texttt{34/items/DifferentialIntegralCalculusVolI/Courant-DifferentialIntegralCalculusVolI.pdf}}
\\[0.2cm]
finden k�nnen.  Schlie�lich enth�lt das Buch von Wrede und Spiegel eine Vielzahl gel�ster
Aufgaben und bietet sich daher besonders zum �ben an.

\section{�berblick �ber die Vorlesung}
Im Rahmen der Vorlesung werden die folgenden Gebiete behandelt:
\begin{enumerate}
\item Im zweiten Kapitel werden die reellen Zahlen mit Hilfe von Dedekind-Schnitten definiert.
\item Das dritte Kapitel f�hrt den Begriff des Grenzwerts f�r Folgen und Reihen ein.
\item Das vierte Kapitel diskutiert die Begriffe Stetigkeit und Differenzierbarkeit.
\item Das f�nfte Kapitel zeigt verschiedene Anwendungen der bis dahin dargestellten Theorie.
      Insbesondere werden \emph{Taylor-Reihen} diskutiert. Diese k�nnen beispielsweise zur
      Berechnung der trigonometrischen Funktionen verwendet werden.  Au�erdem diskutieren wir in
      diesem Kapitel Verfahren zur numerischen L�sung von Gleichungen.
\item Das sechste Kapitel besch�ftigt sich mit der Integralrechnung.
\item Im siebten Kapitel zeigen wir, dass  $\pi$ und $e$ keine rationalen Zahlen sind.
\item Im letzten Kapitel diskutieren wir Fourier-Reihen.
\end{enumerate}

\section{Ziel der Vorlesung}
Wir werden im Rahmen der Vorlesung nicht die Zeit haben, alle Aspekte der Analysis zu besprechen.
Insbesondere werden wir viele interessante Anwendungen der Analysis in der Informatik nicht
diskutieren k�nnen.  Das ist aber auch gar nicht das Ziel dieser Vorlesung:  Mir geht es vor allem
darum, Ihnen die F�higkeit zu vermitteln, sich 
selbstst�ndig in mathematische Fachliteratur hineinarbeiten zu k�nnen.  Dazu m�ssen Sie in der Lage
sein, mathematische Beweise sowohl zu verstehen als auch selber entwickeln zu k�nnen.   Dies ist ein
wesentlicher Unterschied zu der Mathematik, an die sich viele von Ihnen auf der Schule gew�hnthaben:
Dort werden prim�r Verfahren vermitteln, mit denen sich spezielle Probleme l�sen lassen.  Die Kenntnis
solcher Verfahren ist allerdings in der Praxis nicht mehr wichtig, denn heutzutage werden solche 
Verfahren programmiert und daher besteht kein Bedarf mehr daf�r, solche Verfahren von Hand
anzuwenden.  Aus diesem Grund wird in dieser Vorlesung der mathematische Beweis-Begriff im
Vordergrund stehen.  Die Analysis dient uns dabei als ein Beispiel einer mathematischen Theorie, an
Hand derer wir das mathematische Denken �ben k�nnen. 

\section{Notation}
In diesem Skript definieren wir die Menge der nat�rlichen Zahlen $\mathbb{N}$ �ber die Formel
\\[0.2cm]
\hspace*{1.3cm}
$\mathbb{N} := \{ 1, 2, 3, \cdots \}$.
\\[0.2cm]
Im Gegensatz zu der Vorlesung �ber Lineare Algebra im letzten Semester wird die Zahl $0$ in diesem
Skript also \underline{nicht} als nat�rliche Zahl aufgefasst.  Weiter definieren wir
\\[0.2cm]
\hspace*{1.3cm}
$\mathbb{N}_0 := \{ 0 \} \cup \mathbb{N}$.


\section{Eine Bitte}
Diese Skript enth�lt noch eine Menge Tipp-Fehler.  Sollte Ihnen ein Fehler auffallen, so bitte ich
um einen Hinweis unter der Adresse
\\[0.2cm]
\hspace*{1.3cm}
\href{mailto:karl.stroetmann@dhbw-mannheim.de}{karl.stroetmann@dhbw-mannheim.de}.
\\[0.2cm]
Wenn Sie mit \href{http://github.com}{\texttt{github}} vertraut sind, k�nnen Sie mir auch gerne einen
\href{https://help.github.com/articles/using-pull-requests}{\textsl{Pull Request}} schicken.


%%% Local Variables: 
%%% mode: latex
%%% TeX-master: "analysis"
%%% End: 

\chapter{Die Definition der reellen Zahlen}
Bevor wir mit der eigentlichen Analysis beginnen m�ssen wir kl�ren, was genau reelle Zahlen
�berhaupt sind.  Anschaulich werden reelle Zahlen zur Angabe von L�ngen ben�tigt, denn in der Geometrie
reicht es nicht, mit den rationalen Zahlen zu arbeiten.  Das liegt daran, dass die Diagonale eines
Quadrats der Seitenl�nge 1 nach dem 
\href{http://de.wikipedia.org/wiki/Satz_des_Pythagoras}{Satz des Pythagoras} die L�nge $\sqrt{2}$ hat.
Wir hatten im letzten Semester aber gesehen, dass es keine rationale Zahl $r$ gibt,
so dass $r^2 = 2$ ist.  Folglich reichen die rationalen Zahlen nicht aus, alle in der Geometrie
m�glichen L�ngen anzugeben. 

\section{Dedekind-Schnitte}
Bis jetzt haben wir so getan, als w�ssten wir schon, was reelle Zahlen sind.  Aus der Schule bringen  
Sie gewiss eine anschauliche Vorstellung der reellen Zahlen mit, aber diese Vorstellung gilt es  
nun zu formalisieren, denn sonst k�nnen wir den Nachweis der \emph{Vollst�ndigkeit} der reellen  
Zahlen nicht f�hren.   Unter der Vollst�ndigkeit der reellen Zahlen verstehen wir anschaulich die Eigenschaft, 
dass es auf  der reellen Zahlengeraden keine \emph{L�cken} gibt.  

Die zentrale Idee bei der Konstruktion der reellen Zahlen ist die Beobachtung, dass eine reelle
Zahl $x$ durch die Menge $M_1$ aller rationalen Zahlen kleiner als $x$ und die Menge $M_2$ der
rationalen Zahlen gr��er-gleich $x$ bereits vollst�ndig festgelegt wird.  Definieren wir f�r eine
reelle Zahl $x$
\\[0.2cm]
\hspace*{1.3cm}
$M_1 := \{ q \in \mathbb{Q} \mid q < x \}$ \quad und \quad 
$M_2 := \{ q \in \mathbb{Q} \mid x \leq q \}$,
\\[0.2cm]
so liegt $x$ gerade zwischen $M_1$ und $M_2$.  Falls die Menge $M_2$ kein Minimum hat, so haben die
rationalen Zahlen  zwischen $M_1$ und $M_2$ eine L�cke.  Die Idee ist nun, die reellen Zahlen als
die L�cken der rationalen Zahlen zu definieren um dadurch sicherzustellen, dass es bei den
reellen Zahlen keine L�cken mehr gibt.  
Versuchen wir den Begriff eines \emph{L�cke} zu pr�zisieren, so kommen wir zur nun folgende Definition eines
Dedekind'schen-Schnittes.

\begin{Definition}[Dedekind-Schnitt] \lb
Ein Paar $\pair(M_1,M_2)$ hei�t \href{http://de.wikipedia.org/wiki/Dedekindscher_Schnitt}{\emph{Dedekind-Schnitt}}
(\href{http://de.wikipedia.org/wiki/Richard_Dedekind}{\textrm{Richard Dedekind}}, 1831-1916)
falls folgendes gilt:
\begin{enumerate}
\item $M_1 \subseteq  \mathbb{Q}$, \quad $M_2 \subseteq \mathbb{Q}$.
\item $M_1 \not= \emptyset$, \quad $M_2 \not= \emptyset$.
\item $\forall x_1 \in M_1: \forall x_2 \in M_2: x_1 < x_2$.

       Diese Bedingung besagt, dass alle Elemente aus $M_1$ kleiner als alle Elemente aus $M_2$
       sind.  Diese Bedingung bezeichnen wir als die \emph{Trennungs-Eigenschaft}.
\item $M_1 \cup M_2 = \mathbb{Q}$.
\item $M_1$ hat kein Maximum.

      Da alle Elemente aus $M_1$ kleiner als alle Elemente von $M_2$ sind und da dar�ber hinaus 
      $M_2 \not= \emptyset$ ist, ist $M_1$ sicher nach oben beschr�nkt.  Aber wenn f�r ein $y$
      \\[0.2cm]
      \hspace*{1.3cm}
      $\forall x \in M_1: x \leq y$
      \\[0.2cm]
      gilt, dann darf $y$ eben kein Element von $M_1$ sein.  Als Formel schreibt sich das als
      \\[0.2cm]
      \hspace*{1.3cm}
      $\forall y \in Q: \bigl((\forall x \in M_1: x \leq y) \rightarrow y \not\in M_1\bigr)$. \eod
\end{enumerate}
\end{Definition}

\example Definieren wir
\\[0.2cm]
\hspace*{1.3cm} 
$M_1 := \{ x \in \mathbb{Q} \mid x \leq 0 \vee x^2 \leq 2 \}$ \quad und \quad
$M_2 := \{ x \in \mathbb{Q} \mid x > 0 \wedge x^2 > 2 \}$,
\\[0.2cm]
so enth�lt $M_1$ alle die Zahlen, die kleiner oder gleich $\sqrt{2}$ sind, w�hrend
$M_2$ alle Zahlen enth�lt, die gr��er als $\sqrt{2}$ sind. Das Paar $\pair(M_1,M_2)$ ist dann ein
Dedekind-Schnitt.  Intuitiv spezifiziert dieser Dedekind-Schnitt die Zahl $\sqrt{2}$.
\eox

Das Beispiel legt nahe, die Menge der reellen Zahlen formal als die Menge aller Dedekind-Schnitte zu
definieren
\\[0.2cm]
\hspace*{1.3cm}
\colorbox{orange}{
\framebox{
\framebox{
$\mathbb{R} := \bigl\{ \pair(M_1,M_2) \in 2^\mathbb{Q} \times 2^\mathbb{Q} \mid \mbox{$\pair(M_1,M_2)$ ist eine Dedekind-Schnitt}\bigr\}$.}}}
\\[0.2cm]
Nach dieser Definition m�ssen wir nun zeigen, wie sich auf der so definierten Menge der reellen
Zahlen die arithmetischen Operationen Addition, Subtraktion, Multiplikation und Division definieren
lassen und wie die Relation $\leq$ f�r zwei Dedekind-Schnitte definiert
werden kann.  Zus�tzlich m�ssen wir nachweisen, dass unsere Definitionen die Eigenschaften haben,
die wir von diesen Operationen erwarten.

Bei einem Dedekind-Schnitt $\pair(M_1,M_2)$ ist die Menge $M_2$ durch die Angabe von
$M_1$ bereits vollst�ndig festgelegt, denn aus der Gleichung $M_1 \cup M_2 = \mathbb{Q}$ folgt
sofort $M_2 = \mathbb{Q} \backslash M_1$.  Die Frage ist nun, welche Eigenschaften eine Menge $M$
haben muss, damit umgekehrt das Paar $\pair(M, \mathbb{Q} \backslash M)$ ein Dedekind-Schnitt ist.
Die Antwort auf diese Frage wird in der nun folgenden Definition einer \emph{Dedekind-Menge} gegeben.


\begin{Definition}[Dedekind-Menge]
Eine Menge $M \subseteq \mathbb{Q}$ ist eine \emph{Dedekind-Menge} genau dann, wenn die
folgenden Bedingungen erf�llt sind.
\begin{enumerate}
\item $M \not= \emptyset$,
\item $M \not= \mathbb{Q}$,
\item $\forall x, y \in \mathbb{Q}: \bigl(y < x \wedge x \in M \rightarrow y \in M)$.

      Die letzte Bedingung besagt, dass $M$ \emph{nach unten abgeschlossen} ist:  Wenn eine
      Zahl $x$ in $M$ liegt, dann liegt auch jede Zahl, die kleiner als $x$ ist, in $M$.
\item Die Menge $M$ hat kein Maximum, es gibt also kein $m \in M$, so dass
      \\[0.2cm]
      \hspace*{1.3cm}
      $x \leq m$ \quad f�r alle $x \in M$ gilt.
      \\[0.2cm]
      Diese Bedingung k�nnen wir auch etwas anders formulieren:  Wenn $x \in M$ ist, dann finden wir
      immer ein $y \in M$, dass noch gr��er als $x$ ist, denn sonst w�re $x$ ja das Maximum von $M$.
      Formal k�nnen wir das als
      \\[0.2cm]
      \hspace*{1.3cm}
      $\forall x \in M: \exists y \in M: x < y$
      \\[0.2cm]
      schreiben.
\end{enumerate}
\end{Definition}

\exercise
Zeigen Sie, dass eine Menge $M \subseteq \mathbb{Q}$ genau dann eine Dedekind-Menge ist, wenn das Paar $\pair(M,\mathbb{Q} \backslash M)$
ein Dedekind-Schnitt ist. \eox

\solution
Da es sich bei der zu beweisenden Aussage um eine �quivalenz-Aussage handelt,  zerf�llt der Beweis
in zwei Teile.
\begin{enumerate}
\item[``$\Rightarrow$'':] Zun�chst nehmen wir an, dass $M \subseteq \mathbb{Q}$ eine Dedekind-Menge ist.  Wir haben zu
      zeigen, dass dann $\pair(M,\mathbb{Q} \backslash M)$ ein Dedekind-Schnitt ist.  Von den zu
      �berpr�fenden Eigenschaften ist nur die Trennungs-Eigenschaft nicht offensichtlich.
      Sei als $x \in M$ und $y \in \mathbb{Q} \backslash M$.  Wir haben zu zeigen, dass dann
      \\[0.2cm]
      \hspace*{1.3cm}
      $x < y$
      \\[0.2cm]
      gilt.  Wir f�hren diesen Nachweis indirekt und nehmen an, dass $y \leq x$.  Da $M$ nach unten
      abgeschlossen ist, folgt daraus aber $y \in M$, was im Widerspruch zu $y \in \mathbb{Q} \backslash M$ steht.
      Dieser Widerspruch zeigt, dass $x < y$ ist und das war zum Nachweis der Trennungs-Eigenschaft
      zu zeigen.
\item[``$\Leftarrow$'':] Nun nehmen wir an, dass $\pair(M, \mathbb{Q} \backslash M)$ ein Dedekind-Schnitt ist und
     zeigen, dass dann $M$ eine Dedekind-Menge sein muss.  Von den zu �berpr�fenden Eigenschaften
     ist nur Tatsache, dass $M$ nach unten abgeschlossen ist, nicht offensichtlich.  Sei also $x \in M$
     und $y < x$.  Wir haben zu zeigen, dass dann $y$ ebenfalls ein Element von $M$ ist.  Wir f�hren
     diesen Nachweis indirekt und nehmen $y \in \mathbb{Q} \backslash M$ an.  Aufgrund der
     Trennungs-Eigenschaft des Dedekind-Schnitts $\pair(M, \mathbb{Q} \backslash M)$ muss dann
     \\[0.2cm]
     \hspace*{1.3cm}
     $x < y$
     \\[0.2cm]
     gelten, was im Widerspruch zu $y < x$ steht.  Dieser Widerspruch zeigt, dass $y \in M$ gilt
     und das war zu zeigen.  \qed
\end{enumerate}

Die letzte Aufgabe hat gezeigt, dass Dedekind-Schnitte und Dedekind-Mengen zu einander �quivalent
sind.  Daher werden wir im Folgenden mit Dedekind-Mengen
arbeiten,  denn das macht die Notation einfacher.  Wir definieren dazu
$\mathcal{D}$ als die Menge aller rationalen Dedekind-Mengen, wir setzen also
\\[0.2cm]
\hspace*{1.3cm}
\colorbox{orange}{
$\mathcal{D} := \bigl\{ M \in 2^{\mathbb{Q}} \mid \mbox{$M$ is Dedekind-Menge} \bigr\}$}
\\[0.2cm]
und identifizieren $\mathcal{D}$ mit der Menge $\mathbb{R}$.  Die n�chste Aufgabe zeigt, wie wir auf
der Menge $\mathcal{D}$ eine lineare Ordnung definieren k�nnen. 

\exercise
Auf der Menge $\mathcal{D}$ definieren wir eine bin�re Relation $\leq$ durch die Festsetzung
\\[0.2cm]
\hspace*{1.3cm}
$A \leq B \;\stackrel{\mathrm{def}}{\Longleftrightarrow}\; A \subseteq B$ \quad f�r alle $A,B \in \mathcal{D}$.
\\[0.2cm]
Zeigen Sie, dass die so definierte Relation $\leq$ eine lineare Ordnung auf der Menge $\mathcal{D}$ ist.

\solution
Es ist zu zeigen, dass die Relation $\leq$ reflexiv, anti-symmetrisch und transitiv ist und dass au�erdem die Linearit�ts-Eigenschaft
\\[0.2cm]
\hspace*{1.3cm} $A \leq B \vee B \leq A$ \quad f�r alle Dedekind-Mengen $A,B \in \mathcal{D}$
\\[0.2cm]
gilt.  Die Reflexivit�t, Anti-Symmetrie und Transitivit�t der Relation $\leq$ folgen sofort aus der Reflexivit�t,
Anti-Symmetrie und Transitivit�t der Teilmengen-Relation $\subseteq$.  Es bleibt, den Nachweis der
Linearit�ts-Eigenschaft zu f�hren.  Seien also $A,B \in \mathcal{D}$ gegeben.  Falls $A = B$ ist, gilt sowohl
$A \subseteq B$ als auch $B \subseteq A$, woraus sofort $A \leq B$ und $B \leq B$ folgt.  Wir nehmen
daher an, dass $A \not= B$ ist.  Dann gibt es zwei M�glichkeiten:
\begin{enumerate}
\item Fall: Es existiert ein $x \in \mathbb{Q}$ mit $x \in A$ und $x \not\in B$.

            Wir zeigen, dass dann $B \subseteq A$, also $B \leq A$ gilt.  Zum Nachweis der Beziehung
            $B \subseteq A$ nehmen wir an, dass $y \in B$ ist und m�ssen $y \in A$ zeigen.

            Wir behaupten, dass $y < x$ ist und f�hren den Beweis dieser Behauptung indirekt, nehmen also
            $x \leq y$ an.  Da die Dedekind-Menge $B$ nach unten abgeschlossen ist und $y \in B$ ist, w�rde daraus
            \\[0.2cm]
            \hspace*{1.3cm}
            $x \in B$
            \\[0.2cm]
            folgen, was im Widerspruch zu der in diesem Fall gemachten Annahme $x \not\in B$ steht.  Also
            haben wir
            \\[0.2cm]
            \hspace*{1.3cm}
            $y < x$.
            \\[0.2cm]
            Da die Menge $A$ als Dedekind-Menge nach unten abgeschlossen ist und $x \in A$ ist, folgt
            \\[0.2cm]
            \hspace*{1.3cm}
            $y \in A$,
            \\[0.2cm]
            so dass wir $B \subseteq A$ gezeigt haben.
\item Fall: Es existiert ein $x \in \mathbb{Q}$ mit $x \in B$ und $x \not\in A$.

            Dieser Fall ist offenbar analog zum ersten Fall.
            \qed
\end{enumerate}


\begin{Definition}[Infimum]
Eine Menge $\mathcal{M} \subseteq \mathcal{D}$ ist \emph{in $\mathcal{D}$ nach unten beschr�nkt}, falls es ein
 $U \in \mathcal{D}$ gibt, so dass gilt:
\\[0.2cm]
\hspace*{1.3cm}
$\forall A \in \mathcal{M}:  U \leq A$.
\\[0.2cm]
Eine Menge $I \in \mathcal{D}$ ist das \emph{Infimum} einer Menge $\mathcal{M}$, wenn $I$ die gr��te
untere Schranke von $\mathcal{M}$ ist, wenn also 
\\[0.2cm]
\hspace*{1.3cm}
$\forall A \in \mathcal{M}: I \leq A$ \quad \mbox{und} \quad
$\forall T \in \mathcal{D}: \Bigl(\bigl(\forall A \in \mathcal{M}: T \leq A \bigr)\rightarrow T \leq I \Bigr)$
\\[0.2cm]
gilt.  In diesem Fall schreiben wir
\\[0.2cm]
\hspace*{1.3cm}
$S = \inf( \mathcal{M} )$.
\eod
\end{Definition}

\begin{Definition}[Vollst�ndige Ordnung]
  Ein Paar $\pair(M, \leq)$ bestehend aus einer Menge $M$ und einer Relation $\leq \subseteq M \times M$ 
  ist eine \emph{vollst�ndige} Ordnung genau denn, wenn folgendes gilt:
  \begin{enumerate}
  \item Die Relation $\leq$ ist reflexiv auf $M$, es gilt 
        \\[0.2cm]
        \hspace*{1.3cm}
        $\forall x \in M: x \leq x$.
  \item Die Relation $\leq$ ist anti-symmetrisch, es gilt
        \\[0.2cm]
        \hspace*{1.3cm}
        $\forall x, y \in M: x \leq y \wedge y \leq x \rightarrow x = y$.
  \item Die Relation $\leq$ ist transitiv, es gilt
        \\[0.2cm]
        \hspace*{1.3cm}
        $\forall x, y, z \in M: x \leq y \wedge y \leq z \rightarrow x \leq z$.
  \item Die Relation $\leq$ ist linear, es gilt
        \\[0.2cm]
        \hspace*{1.3cm}
        $\forall x, y \in M: x \leq y \vee y \leq x$.
  \item Zu jeder nicht-leeren und nach unten beschr�nkte Menge $X \subseteq M$ exxistiert ein
        Infimum in $M$. \eod
  \end{enumerate}
\end{Definition}
Die ersten drei Eigenschaften der Definition einer vollst�ndigen Ordnung fordern, dass 
$\pair(M, \leq)$ eine partielle Ordnung auf $M$ ist, die ersten vier Eigenschaften fordern, dass 
$\pair(M, \leq)$ eine lineare Ordnung auf $M$ ist und die f�nfte Eigenschaft fordert schlie�lich die
Vollst�ndigkeit der Ordnung.


\remark
Die Menge der rationalen Zahlen zusammen mit der �blichen $\leq$-Relation ist
\underline{nicht} vollst�ndig, denn beispielsweise ist die Menge
\\[0.2cm]
\hspace*{1.3cm}
$M := \{ x \in \mathbb{Q} \mid x \geq 0 \wedge x^2 \geq 2 \}$
\\[0.2cm]
nach unten beschr�nkt, aber innerhalb der rationalen Zahlen hat die Menge $M$ kein Infimum, denn
$\sqrt{2}$ ist keine rationale Zahl.  
In der n�chsten Aufgabe werden wir sehen,  dass die reellen Zahlen zusammen mit
der oben definierten Ordnung $\leq$ vollst�ndig sind.

\exercise
Zeigen Sie, dass jede nicht-leere und in $\mathcal{D}$ nach unten beschr�nkte Menge
$\mathcal{M} \subseteq \mathcal{D}$ ein Infimum $I \in \mathcal{D}$ hat.  \eox


\solution
Bevor wir das gesuchte Infimum konstruieren, gilt es eine technische Schwierigkeit zu �berwinden,
die aus der Forderung entsteht, dass Dedekind-Mengen kein Maximum besitzen.  Wir
definieren f�r eine beliebige Teilmenge $A \subseteq \mathbb{Q}$
die Menge $A^-$, die in dem Falle, dass die Menge $A$ kein Maximum hat, mit der Menge $A$ identisch ist.
Andernfalls wird bei der Definition von $A^-$ das Maximum aus der Menge $A$ entfernt:
\\[0.2cm]
\hspace*{1.3cm}
$A^- := \left\{
\begin{array}[c]{ll}
  A \backslash \{ \max(A) \} & \mbox{falls die Menge $A$ ein Maximum hat}; \\
  A                          & \mbox{sonst}.
\end{array}
\right.$
\\[0.2cm]
Nun definieren wir zu der gegebenen Menge $\mathcal{M} \subseteq \mathcal{D}$ eine Menge $J$ als
\\[0.2cm]
\hspace*{1.3cm}
$J := \{ q \in \mathbb{Q} \mid \forall A \in \mathcal{M}: q \in A \} = \bigcap \mathcal{M}$ 
\quad und setzen $I := J^-$. 
\\[0.2cm]
Wir zeigen, dass die so definierte Menge $I$ eine Dedekind-Menge ist.
\begin{enumerate}
\item Wir zeigen $I \not= \{\}$.

      Nach Voraussetzung ist die Menge $\mathcal{M}$ nach unten beschr�nkt.  Also gibt es eine Dedekind-Menge 
      $U \in \mathcal{D}$, so dass
      \\[0.2cm]
      \hspace*{1.3cm}
      $U \subseteq A$ \quad f�r alle $A \in \mathcal{M}$
      \\[0.2cm]
      gilt.  Wegen $J = \bigcap \mathcal{M}$ folgt dann $U \subseteq J$.  Nun gibt es zwei
      M�glichkeiten:
      \begin{enumerate}
      \item $I = J$.

            Damit gilt auch $U \subseteq I$  und da $U$ als Dedekind-Menge nicht leer
            ist, folgt $I \not= \{\}$.
      \item $I = J \backslash \{ \max(J) \}$, die Menge $J$ hat also ein Maximum.
            Wegen $U \subseteq J$ ist $\max(J)$ dann sicher auch eine obere Schranke
            von $U$.  W�re $\max(J) \in U$, dann w�re $\max(J)$ auch das Maximum von $U$,
            was nicht sein kann, da $U$ als Dedekind-Menge kein Maximum hat.
            Folglich gilt $\max(J) \not\in U$ und damit folgt, dass
            \\[0.2cm]
            \hspace*{1.3cm}
            $U \subseteq J \backslash \{ \max(J) \} = I$
            \\[0.2cm]
            gilt.  Da $U \not= \emptyset$ ist, folgt auch $I \not= \emptyset$.
      \end{enumerate}
\item Wir zeigen $I \not= \mathbb{Q}$.

      Nach Voraussetzung ist die Menge $\mathcal{M}$ nicht leer.  Sei $A \in \mathcal{M}$. Nach
      Definition von $J$ gilt $J \subseteq A$ und wegen $I \subseteq J$ folgt  $I \subseteq A$.
      Da $A$ eine Dedekind-Menge ist, gilt $A \not= \mathbb{Q}$.  Wegen $I \subseteq A$ folgt daraus sofort
      $I \not= \mathbb{Q}$.
\item Wir zeigen, dass $I$ nach unten abgeschlossen ist.

      Es sei also $x \in I$ und $y < x$.  Dann gilt nat�rlich auch $x \in J$ und damit haben wir
      \\[0.2cm]
      \hspace*{1.3cm}
      $x \in A$ \quad f�r alle $A \in \mathcal{M}$.
      \\[0.2cm]
      Nun sind die Mengen $A \in \mathcal{M}$ alle Dedekind-Mengen und damit nach unten abgeschlossen.
      Damit folgt aus $x \in A$ und $y < x$ also
      \\[0.2cm]
      \hspace*{1.3cm}
      $y \in A$ \quad f�r alle $A \in \mathcal{M}$.      
      \\[0.2cm]
      Nach Definition der Menge $J$ folgt daraus 
      \\[0.2cm]
      \hspace*{1.3cm}
      $y \in J$.
      \\[0.2cm]
      Aus $x \in J$ und $y < x$ folgt, dass $y$ sicher nicht das Maximum der Menge $J$ ist.  Damit gilt
      dann auch
      \\[0.2cm]
      \hspace*{1.3cm}
      $y \in I$
      \\[0.2cm]
      und das war zu zeigen.
\item Wir zeigen, dass die Menge $I$ kein Maximum hat.
      Wir f�hren dazu eine Fallunterscheidung nach der Definition von $I = J^-$ durch.
      \begin{enumerate}
      \item Fall: $I = J$.

            Dann hat die Menge $J$ kein Maximum und damit hat auch $I$ kein Maximum.
      \item Fall: $I = J \backslash \{ a \}$ mit $a = \max(J)$.
            
            Wir f�hren den Nachweis indirekt und nehmen an, dass es ein $b \in I$ mit $b = \max(I)$ gibt.
            Aus $b \in J$,  und $a = \max(J)$ folgt dann $b < a$, denn der Fall $b = a$ scheidet
            wegen $a \not\in I$ aus.  Das arithmetische Mittel von $a$ und $b$ liegt zwischen $a$ und
            $b$, wir haben also
            \\[0.2cm]
            \hspace*{1.3cm}
            $b < \bruch{1}{2} \cdot (a + b) < a$.
            \\[0.2cm]
            Da die Menge $J$ nach unten abgeschlossen ist und $a \in J$ ist, folgt dann
            \\[0.2cm]
            \hspace*{1.3cm}
            $\bruch{1}{2} \cdot (a + b) \in J$
            \\[0.2cm]
            und wegen $\bruch{1}{2} \cdot (a + b) \not= a$ muss dann auch
            \\[0.2cm]
            \hspace*{1.3cm}
            $\bruch{1}{2} \cdot (a + b) \in I$
            \\[0.2cm]
            gelten, denn die Mengen $I$ und $J$ stimmen ja bis auf das Element $a$ �berein.  Wegen
            \\[0.2cm]
            \hspace*{1.3cm}
            $b < \bruch{1}{2} \cdot (a + b)$.
            \\[0.2cm]
            steht dies aber im Widerspruch zu $b = \max(I)$, so dass die Annahme, dass 
            es ein $b \in I$ mit $b = \max(I)$ gibt, widerlegt ist.  
      \end{enumerate}
\end{enumerate}
Um den Beweis abzuschlie�en zeigen wir nun, dass $I$ tats�chlich die gr��te untere Schranke der Menge
$\mathcal{M}$ ist.  Aus der Definition
\\[0.2cm]
\hspace*{1.3cm}
$J := \bigcap \mathcal{M}$
\\[0.2cm]
folgt zun�chst $J \subseteq A$ f�r alle $A \in M$.  Wegen $I \subseteq J$ haben wir dann
\\[0.2cm]
\hspace*{1.3cm}
$I \subseteq A$ \quad f�r alle $A \in \mathcal{M}$
\\[0.2cm]
und damit ist gezeigt, dass $I$ eine untere Schranke der Menge $\mathcal{M}$ ist.  Es bleibt zu zeigen,
dass $I$ die gr��te untere Schranke der Menge $\mathcal{M}$ ist.  Sei also $U \in \mathcal{D}$ eine
weitere untere Schranke von $\mathcal{M}$. Wir m�ssen zeigen, dass dann $U \subseteq I$ gilt.  Die
Aussage, dass $U$ eine untere Schranke der Menge $\mathcal{M}$ bedeutet, dass
\\[0.2cm]
\hspace*{1.3cm}
$U \subseteq A$ \quad f�r alle $A \in \mathcal{M}$ gilt.
\\[0.2cm]
Sei nun $x \in U$.  Dann gilt also auch
\\[0.2cm]
\hspace*{1.3cm}
$x \in A$ \quad f�r alle $A \in \mathcal{M}$.
\\[0.2cm]
und aus der Definition $J = \bigcap \mathcal{M}$ folgt $x \in J$.  Damit haben wir schon mal 
\\[0.2cm]
\hspace*{1.3cm}
$U \subseteq J$
\\[0.2cm]
gezeigt.  Um auch die Ungleichung $U \subseteq I$ zu zeigen,  nehmen wir an, dass zus�tzlich zu $x \in U$
auch $x = \max(J)$ gilt.  Aus $U \subseteq J$ folgt dann auch $x = \max(U)$,  was aber nicht sein kann,
denn $U$ ist eine Dedekind-Menge und enth�lt daher kein Maximum.  Daher wird $x$ beim �bergang von $J$ zu
$I = J^-$ nicht aus der Menge $J$ entfernt und es gilt auch $x \in I$.   Insgesamt haben wir damit
$U \subseteq I$
gezeigt und damit ist $I$ tats�chlich die gr��te untere Schranke der Menge $\mathcal{M}$.
\qed



\begin{Definition}[Supremum]
Eine Menge $\mathcal{M} \subseteq \mathcal{D}$ ist \emph{in $\mathcal{D}$ nach oben beschr�nkt}, falls es ein
 $O \in \mathcal{D}$ gibt, so dass gilt:
\\[0.2cm]
\hspace*{1.3cm}
$\forall A \in \mathcal{M}:  A \leq O$.
\\[0.2cm]
Eine Menge $S \in \mathcal{D}$ ist das \emph{Supremum} einer Menge $\mathcal{M}$, wenn $S$ die kleinste
obere Schranke von $\mathcal{M}$ ist, wenn also 
\\[0.2cm]
\hspace*{1.3cm}
$\forall A \in \mathcal{M}: A \leq S$ \quad \mbox{und} \quad
$\forall T \in \mathcal{D}: \Bigl(\bigl(\forall A \in \mathcal{M}: A \leq T \bigr)\rightarrow S \leq T \Bigr)$
\\[0.2cm]
gilt.  In diesem Fall schreiben wir
\\[0.2cm]
\hspace*{1.3cm}
$S = \sup( \mathcal{M} )$.
\eod
\end{Definition}


\exercise
Zeigen Sie, dass jede nicht-leere und in $\mathcal{D}$ oben beschr�nkte Menge
$\mathcal{M} \subseteq \mathcal{D}$ ein Supremum $S \in \mathcal{D}$ hat.
\vspace*{0.2cm}

\noindent
\textbf{Hinweis}:  Es gibt zwei M�glichkeiten, diesen Nachweis zu f�hren.
\begin{enumerate}
\item Wir definieren $S := \bigcup \mathcal{M}$.  Es l�sst sich zeigen, dass die so definierte Menge
      $S$ bereits eine Dedekind-Menge ist und damit insbesondere kein Maximum enth�lt.  
      Weiter k�nnen Sie dann nachweisen, dass tats�chlich $S = \sup(\mathcal{M})$ gilt.  
\item Alternativ k�nnen Sie auch versuchen auszunutzen, dass wir bereits gezeigt haben, dass f�r
      eine nach unten beschr�nkte nicht-leere Menge $\mathcal{M} \subseteq \mathcal{D}$
      das Infimum existiert.

      Diese zweite Alternative ist deutlich einfacher als die erste, aber die erste Alternative hat den
      Vorteil, dass Sie damit einen unabh�ngigen Beweis der Vollst�ndigkeit der reellen Zahlen erbringen.
      \eox 
\end{enumerate}

\begin{Definition}[Addition von Dedekind-Mengen]
Es seien $A$ und $B$ Dedekind-Mengen.  Dann definieren wir die Summe $A + B$ wie folgt:
\\[0.2cm]
\hspace*{1.3cm}
$A + B := \{ x + y \mid x \in A \wedge y \in B \}$. \eod
\end{Definition}

\exercise
Es seien $A,B \in \mathcal{D}$.  Zeigen Sie, dass dann auch $A + B \in \mathcal{D}$ ist.
\eox

\solution
Wir zeigen, dass $A + B$ eine Dedekind-Menge ist.
\begin{enumerate}
\item Wir zeigen $A + B \not= \{\}$.

      Da $A$ eine Dedekind-Menge ist, gibt es ein Element $a \in A$ und da $B$ ebenfalls eine
      Dedekind-Menge ist, gibt es auch ein Element $b \in B$.  Nach Definition von $A + B$ folgt dann
      $a + b \in A + B$ und damit gilt $A + B \not= \{\}$.
\item Wir zeigen $A + B \not= \mathbb{Q}$.

      Da $A$ und $B$ Dedekind-Mengen sind, gilt $A \not= \mathbb{Q}$ und $B \not= \mathbb{Q}$.  Also gibt es 
      $x,y \in \mathbb{Q}$ mit $x \not\in A$ und $y \not\in B$.  Wir zeigen, dass dann  $x + y \not\in A + B$
      ist und f�hren diesen Nachweis indirekt.  Wir nehmen also an, dass
      \\[0.2cm]
      \hspace*{1.3cm}
      $x + y \in A + B$
      \\[0.2cm]
      gilt.  Nach Definition der Menge $A + B$ gibt es dann ein $a \in A$ und ein $b \in B$, so dass
      \\[0.2cm]
      \hspace*{1.3cm}
      $x + y = a + b$
      \\[0.2cm]
      ist.  Aus $x \not\in A$ und $a \in A$ folgt, dass 
      \\[0.2cm]
      \hspace*{1.3cm}
      $a < x$ 
      \\[0.2cm]
      ist, denn da $A$ eine Dedekind-Menge ist, w�rde aus $x \leq a$ sofort $x \in A$ folgern.  Weil $B$ eine
      Dedekind-Menge ist, gilt dann auch
      \\[0.2cm]
      \hspace*{1.3cm}
      $b < y$.
      \\[0.2cm]
      Addieren wir diese beiden Ungleichungen, so erhalten wir
      \\[0.2cm]
      \hspace*{1.3cm}
      $a + b < x + y$,
      \\[0.2cm]
      was im Widerspruch zu der Gleichung $x + y = a + b$ steht.
\item Wir zeigen, dass die Menge $A + B$ nach unten abgeschlossen ist.

      Sei also $x \in A + B$ und $y < x$.  Nach Definition von $A + B$ gibt es dann ein $a \in A$ und ein
      $b \in B$, so dass $x = a + b$ gilt.  Wir definieren
      \\[0.2cm]
      \hspace*{1.3cm} 
      $c := x - y$, \quad $u := a - \bruch{1}{2} \cdot c$ \quad und \quad
      $v := b - \bruch{1}{2} \cdot c$.
      \\[0.2cm]
      Aus $y < x $ folgt zun�chst $c > 0$ und daher gilt $u < a$ und $v < b$.  Da $a \in A$ ist und die
      Menge $A$ als Dedekind-Menge nach unten abgeschlossen ist, folgt $u \in A$.  Analog sehen wir, dass
      auch $v \in B$ ist.  Insgesamt folgt dann
      \\[0.2cm]
      \hspace*{1.3cm}
      $u + v \in A + B$.
      \\[0.2cm]
      Wir haben aber
      \\[0.2cm]
      \hspace*{1.3cm}
      $
      \begin{array}[t]{lcll}
        u + v & = & a - \bruch{1}{2} \cdot c + b - \bruch{1}{2} \cdot c \\[0.2cm]
              & = & a + b - c                                           \\[0.2cm]
              & = & a + b - (x - y)                                     
                  & \mbox{denn $c = x - y$}                             \\[0.2cm]
              & = & x - (x - y)                                     
                  & \mbox{denn $x = a + b$}                             \\[0.2cm]
              & = & y
      \end{array}
      $
      \\[0.2cm]
      Wegen $u + v \in A + B$ haben wir insgesamt $y \in A + B$ nachgewiesen, was zu zeigen war.
\item Wir zeigen, dass die Menge $A + B$ kein Maximum enth�lt.

      Wir f�hren den Beweis indirekt und nehmen an, dass ein $m \in A + B$ existiert, so dass
      $m = \max(A + B)$ gilt.  Nach Definition der Menge $A + B$ gibt es dann ein $a \in A$ und ein 
      $b \in B$ so dass $m = a + b$ ist.  Sei nun $u \in A$.  Wir wollen zeigen, dass $u \leq a$
      ist. W�re $u > a$, dann w�rde auch 
      \\[0.2cm]
      \hspace*{1.3cm}
      $u + b > a + b$
      \\[0.2cm]
      gelten, und da $u + b \in A + B$ ist, k�nnte $m$ dann nicht das Maximum der Menge $A + B$ sein.
      Also gilt $u \leq a$.  Dann ist aber $a$ das Maximum der Menge $A$ und au�erdem in $A$ enthalten.
      Dies ist ein Widerspruch zu der Voraussetzung, dass $A$ eine Dedekind-Menge ist. \qed
\end{enumerate}

\exercise
Zeigen Sie, dass die Menge 
\\[0.2cm]
\hspace*{1.3cm}
$O := \{ x \in \mathbb{Q} \mid x < 0 \}$
\\[0.2cm]
eine Dedekind-Menge ist und zeigen Sie weiter, dass die Struktur $\langle \mathcal{D}, 0, + \rangle$
eine kommutative Gruppe ist.
\eox

\solution
Wir zeigen zun�chst, dass $O$ eine Dedekind-Menge ist und weisen dazu die einzelnen Eigenschaften einer
Dedekind-Menge nach. 
\begin{enumerate}
\item $O \not= \{\}$, denn es gilt $-1 \in O$.
\item $O \not= \mathbb{Q}$, denn es gilt $1 \not\in O$.
\item Die Menge $O$ ist nach unten abgeschlossen.

      Sei $x \in O$ und $y < x$.  Nach Definition von $O$ haben wir
      $x < 0$ und aus $y < x$ und $x < 0$ folgt $y < 0$, also gilt nach Definition der Menge $O$ 
      auch $y \in O$.
\item Die Menge $O$ enth�lt kein Maximum, denn falls $m$ das Maximum der Menge $O$ w�re,
      dann w�re $m < 0$ und daraus folgt sofort $\bruch{1}{2} \cdot m < 0$.  Damit w�re dann nach
      Definition der Menge $O$ auch
      \\[0.2cm]
      \hspace*{1.3cm}
      $\bruch{1}{2} \cdot m \in O$.
      \\[0.2cm]
      Da andererseits aber
      \\[0.2cm]
      \hspace*{1.3cm}
      $m < \bruch{1}{2} \cdot m$
      \\[0.2cm]
      ist, kann dann $m$ nicht das Maximum der Menge $O$ sein.   Folglich hat die Menge $O$ kein Maximum.
\end{enumerate}
\renewcommand{\labelenumi}{\arabic{enumi}.}
Als n�chstes zeigen wir, dass die Menge $O$ das links-neutrale Element bez�glich der Addition von
Dedekind-Mengen ist, wir zeigen also, dass
\\[0.2cm]
\hspace*{1.3cm}
$O + A = A$
\\[0.2cm]
gilt.  Wir spalten den Nachweis dieser Mengen-Gleichheit in den Nachweis zweier Inklusionen auf.
\begin{enumerate}
\item ``$\subseteq$'': Es sei $u \in O + A$.  Wir m�ssen $u \in A$ zeigen.

      Nach Definition von $O + A$ existiert ein $o \in O$
      und ein $a \in A$ mit $u = o + a$.  Aus $o \in O$ folgt $o < 0$.  Also haben wir
      \\[0.2cm]
      \hspace*{1.3cm}
      $u < a$
      \\[0.2cm]
      und da $A$ als Dedekind-Menge nach unten abgeschlossen ist, folgt $u \in A$.
\item ``$\supseteq$'': Sei nun $a \in A$.  Zu zeigen ist $a \in O + A$.

      Da die Menge $A$ eine Dedekind-Menge ist, kann $a$ nicht das Maximum der Menge $A$ sein.
      Folglich gibt es ein $b \in A$, dass gr��er als $a$ ist, wir haben also
      \\[0.2cm]
      \hspace*{1.3cm}
      $a < b$.
      \\[0.2cm]
      Wir definieren $u := a - b$.  Aus $a < b$ folgt dann $u < 0$ und damit gilt $u \in O$.
      Damit haben wir
      \\[0.2cm]
      \hspace*{1.3cm}
      $u + b \in O + A$.
      \\[0.2cm]
      Andererseits gilt
      \\[0.2cm]
      \hspace*{1.3cm}
      $u + b = (a - b) + b = a$,
      \\[0.2cm]
      so dass wir insgesamt $a \in O + A$ gezeigt haben.
\end{enumerate}
Die Tatsache, dass f�r die Addition von Dedekind-Mengen sowohl das Kommutativ-Gesetz als auch das
Assoziativ-Gesetz gilt, folgt unmittelbar aus der Kommutativit�t und der Assoziativit�t der Addition
rationaler Zahlen.

Als n�chstes geben wir f�r eine Dedekind-Menge $A$ das additive Inverse $-\!A$ an:  
\\[0.2cm]
\hspace*{1.3cm}
$-\!A := \{ x \in \mathbb{Q} \mid \exists r \in \mathbb{Q}: r > 0 \wedge -x - r \not\in A \}$.
\\[0.2cm]
Die Menge $-\!A$ enth�lt also die rationalen Zahlen $x$ f�r die $-x$ so gro� ist, dass f�r ein
geeignetes $r > 0$ die Zahl  $-x - r$ kein Element von $A$ mehr ist.  Als erstes zeigen wir, dass 
$-\!A$ eine Dedekind-Menge ist.
\begin{enumerate}
\item Wir zeigen $-\!A \not= \emptyset$.

      Da $A$ eine Dedekind-Menge ist, ist $A \not= \mathbb{Q}$.  Daher gibt es ein $y \in \mathbb{Q}$ 
      mit $y \not\in A$.  Wir definieren 
      \\[0.2cm]
      \hspace*{1.3cm}
      $x := -(y +1)$ \quad und \quad $r := 1$.
      \\[0.2cm]
      Dann gilt offenbar 
      \\[0.2cm]
      \hspace*{1.3cm}
      $-x - r = y + 1 - 1 = y \not\in A$
      \\[0.2cm]
      und nach Definition der Menge $-\!A$ folgt $x \in -\!A$.  Also haben wir $-\!A \not= \emptyset$ gezeigt.
\item Wir zeigen $-\!A \not= \mathbb{Q}$.

      Da $A$ eine Dedekind-Menge ist, gilt $A \not= \emptyset$.  Also gibt es ein $y \in A$.
      Wir definieren
      \\[0.2cm]
      \hspace*{1.3cm}
      $x := -y$.
      \\[0.2cm]
      F�r beliebige $r \in \mathbb{Q}$ mit $r > 0$ gilt dann
      \\[0.2cm]
      \hspace*{1.3cm}
      $-x - r = y - r < y$
      \\[0.2cm]
      und das $y \in A$ ist und $A$ als Dedekind-Menge nach unten abgeschlossen ist, folgt daraus
      \\[0.2cm]
      \hspace*{1.3cm}
      $-x - r \in A$  \quad f�r alle $r \in \mathbb{Q}$ mit $r > 0$.
      \\[0.2cm]
      Nach der Definition von $-\!A$ folgt nun, dass $x$ kein Element von $-\!A$ ist.
      Also gilt $-\!A \not= \mathbb{Q}$.
\item Wir zeigen, dass die Menge $-\!A$ nach unten abgeschlossen ist.

      Sei als $x \in -\!A$ und $y < x$.  Wir m�ssen zeigen, dass dann auch $y \in -\!A$ ist.
      Aus der Voraussetzung  $x \in -\!A$ folgt, dass es ein $r \in \mathbb{Q}$ mit $r>0$ gibt,
      so dass 
      \\[0.2cm]
      \hspace*{1.3cm}
      $-x - r \not\in A$ 
      \\[0.2cm]
      ist.  Aus  $y < x$ folgt $-y > -x$ und damit gilt auch
      \\[0.2cm]
      \hspace*{1.3cm}
      $-x - r < -y - r$.
      \\[0.2cm]
      Wir zeigen, dass $-y - r \not\in A$ ist und f�hren diesen Nachweis indirekt:  W�re 
      $-y - r \in A$, so folgt aus der Ungleichung $-x - r < -y - r$ und der Tatsache, dass $A$ als
      Dedekind-Menge nach unten abgeschlossen ist, dass $-x - r \in A$ w�re, was falsch ist.
      Also folgt $-y - r \not\in A$ und nach Definition von $-\!A$ folgern wir $y \in -\!A$.
      Damit haben wir gezeigt, dass $-\!A$ nach unten abgeschlossen ist.
\item Wir zeigen, dass $-\!A$ kein Maximum hat.

      Wir f�hren diesen Nachweis indirekt und nehmen an, dass $x = \max(-\!A)$ ist.
      Insbesondere ist $x$ dann ein Element von $-\!A$ und daher gibt es dann ein $r \in \mathbb{Q}$
      mit $r>0$ und $-x - r \not\in A$.  Die letzte Formel k�nnen wir auch als
      \\[0.2cm]
      \hspace*{1.3cm}
      $-x - \frac{1}{2} \cdot r - \frac{1}{2} \cdot r \not\in A$
      \\[0.2cm]
      schreiben, woraus wir folgern k�nnen, dass
      \\[0.2cm]
      \hspace*{1.3cm}
      $x + \frac{1}{2} \cdot r \in -\!A$
      \\[0.2cm]
      ist.  Da andererseits $x < x + \frac{1}{2} \cdot r$ ist, kann dann aber $x$ nicht das Maximum
      von $-\!A$ sein.  Dieser Widerspruch zeigt, dass die Menge $-\!A$ kein Maximum hat.
\end{enumerate}
Als n�chstes zeigen wir, dass f�r jede Dedekind-Menge $A$ die Gleichung
\\[0.2cm]
\hspace*{1.3cm}
$(-\!A) + A = O$ 
\\[0.2cm]
gilt.  Wir spalten den Nachweis dieser Mengengleicheit in zwei Teile auf.
\begin{enumerate}
\item ``$\subseteq$'':  Es sei $x + y \in -\!A + A$, also $x \in -\!A$ und $y \in A$.
      Wir haben zu zeigen, dass $x + y \in O$ ist.

      Wegen $x \in -\!A$ gibt es nach Definition der Menge $-\!A$ ein $r \in \mathbb{Q}$ mit $r > 0$, so dass 
      $-x - r \not\in A$ ist.  Da $y \in A$ ist, muss $y < -x - r$ gelten.  Daraus folgt
      \\[0.2cm]
      \hspace*{1.3cm}
      $x + y < -r < 0$
      \\[0.2cm]
      und damit gilt $x + y \in O$.

\item ``$\supseteq$'': Es sei nun $o \in O$, also $o < 0$.  Wir m�ssen ein $x \in -\!A$ und ein $y \in A$
      finden, so dass $o = x + y$ gilt.

      Wir definieren 
      \\[0.2cm]
      \hspace*{1.3cm}
      $r := -\frac{1}{2} \cdot o$.
      \\[0.2cm]
      Da $o < 0$ ist, folgt $r > 0$ und au�erdem gilt $r \in \mathbb{Q}$.  Wir definieren die Menge
      $M$ als
      \\[0.2cm]
      \hspace*{1.3cm}
      $M := \{ n \in \mathbb{Z} \mid n \cdot r \in  A \}$
      \\[0.2cm]
      Da $A \not= \mathbb{Q}$ ist, gibt es ein $z \in \mathbb{Q}$ so dass $z \not\in A$ ist.
      F�r die Zahlen $n \in \mathbb{Z}$, f�r die $n \cdot r > z$ ist, folgt dann $n \not\in M$.
      Folglich ist die Menge $M$ nach oben beschr�nkt und hat daher ein Maximum.  Wir definieren
      \\[0.2cm]
      \hspace*{1.3cm}
      $\widehat{n} := \max(M)$.
      \\[0.2cm]
      Dann gilt $\widehat{n} + 1 \not\in M$, also 
      \\[0.2cm]
      \hspace*{1.3cm}
      $(\widehat{n} + 1) \cdot r \not\in A$.  
      \\[0.2cm]
      Wir  definieren  jetzt
      \\[0.2cm]
      \hspace*{1.3cm}
      $x := \widehat{n} \cdot r$ \quad und \quad $y := -(\widehat{n} + 2) \cdot r$.  
      \\[0.2cm]
      Nach  Definition von $\widehat{n}$ und $M$ gilt dann $x \in A$ und aus 
      $(\widehat{n} + 1) \cdot r \not\in A$ folgt
      \\[0.2cm]
      \hspace*{1.3cm}
      $-y - r = (\widehat{n} + 2) \cdot r - r = (\widehat{n} + 1) \cdot r \not\in A$,
      \\[0.2cm]
      so dass $y \in -\!A$ ist.  Au�erdem gilt
      \\[0.2cm]
      \hspace*{1.3cm}
      $x + y = \widehat{n} \cdot r - (\widehat{n} + 2) \cdot r = - 2 \cdot r = o$.
      \\[0.2cm]
      Damit haben wir $O \subseteq A + -\!A$ gezeigt. 
      \qed
\end{enumerate}


\exercise
�berlegen Sie, wie sich auf der Menge $\mathcal{D}$ eine Multiplikation definieren l�sst,
so dass $\mathcal{D}$ mit dieser Multiplikation und der oben definierten Addition ein K�rper wird.


\solution
Wir nennen eine Dedekind-Menge $A$ positiv, wenn $0 \in A$ gilt.  F�r zwei positive Dedekind-Mengen $A$
und $B$ l��t sich die Multiplikation $A \cdot B$ als
\\[0.2cm]
\hspace*{1.3cm}
$A \cdot B := \{ x \cdot y \mid x \in A \wedge y \in B \wedge x > 0 \wedge y > 0 \} \cup 
              \{ z \in \mathbb{Q} \mid z \leq 0 \}$
\\[0.2cm]
definieren.  Wir zeigen, dass die so definerte Menge $A \cdot B$ eine Dedekind-Menge ist.
Dazu weisen wir die einzelnen Eigenschaften getrennt nach.
\begin{enumerate}
\item Wir zeigen $A \cdot B \not= \{\}$.

      Nach Definition von $A \cdot B$ gilt $0 \in A \cdot B$.   Daraus folgt sofort $A \cdot B \not= \{\}$.
\item Wir zeigen $A \cdot B \not= \mathbb{Q}$.

      Da $A$ und $B$ als Dedekind-Mengen von der Menge $\mathbb{Q}$ verschieden sind, gibt es 
      $u,v \in \mathbb{Q}$ mit $u \not\in A$ und $v \not\in B$. Wir definieren $w := \max(u, v)$.  Dann gilt 
      \\[0.2cm]
      \hspace*{1.3cm}
      $(\forall x \in A: x < w) \wedge (\forall y \in B: y < w)$
      \\[0.2cm]
      Daraus folgt sofort, dass f�r alle $x \in A$ und $y \in B$ die Ungleichung
      \\[0.2cm]
      \hspace*{1.3cm}
      $x \cdot y < w \cdot w$
      \\[0.2cm]
      gilt.  Das hei�t aber $w^2 \not\in A \cdot B$.
\item Wir zeigen, dass $A \cdot B$ nach unten abgeschlossen ist.

      Es sei $x \cdot y \in A \cdot B$ und $z \in \mathbb{Q}$ mit $z < x \cdot y$.  
      Wir m�ssen $z \in A \cdot B$ zeigen.  
      Wir f�hren eine Fall-Unterscheidung danach durch, ob $z > 0$ ist.
      \begin{enumerate}
      \item Fall: $z > 0$.  Dann definieren wir
            \\[0.2cm]
            \hspace*{1.3cm}
            $\alpha := \bruch{z}{x \cdot y}$
            \\[0.2cm]
            Aus $z < x \cdot y$ folgt $\alpha < 1$.  Wir setzen $u := \alpha \cdot x$
            und folglich gilt $u < x$.  Da $A$ nach unten abgeschlossen ist, folgt $u \in A$.
            Damit haben wir insgesamt $u \cdot y \in A \cdot B$.  Es gilt aber
            \\[0.2cm]
            \hspace*{1.3cm}
            $u \cdot y = \alpha \cdot x \cdot y = \bruch{z}{x \cdot y} \cdot x \cdot y = z$,
            \\[0.2cm]
            so dass wir insgesamt $z \in A \cdot B$ gezeigt haben.
      \item Fall: $z \leq 0$.  Dann folgt unmittelbar aus der Definition von $A \cdot B$, dass 
            $z \in A \cdot B$  ist.
      \end{enumerate}
\item Wir zeigen, dass $A \cdot B$ kein Maximum hat.  

      Wir f�hren den Nachweis indirekt und nehmen an, dass die Menge $A \cdot B$ eine Maximum $c$ hat.
      Es gilt dann 
      \\[0.2cm]
      \hspace*{1.3cm}
      $c \in A \cdot B$ \quad und \quad $\forall z \in A \cdot B: z \leq c$.
      \\[0.2cm]
      Nach Definition von $A \cdot B$ gibt es dann ein $a \in A$ und ein $b \in B$ mit $c = a \cdot b$.
      Wir zeigen, dass dann $a$ das Maximum der Menge $A$ ist.  Sei also $u \in A$.  Dann gilt
      \\[0.2cm]
      \hspace*{1.3cm}
      $u \cdot b \in A \cdot B$ \quad und folglich gilt \quad $u \cdot b \leq c = a \cdot b$.
      \\[0.2cm]
      Teilen wir die letzte Ungleichung durch $b$ so folgt
      \\[0.2cm]
      \hspace*{1.3cm}
      $u \leq a$
      \\[0.2cm]
      und damit w�re $a$ das Maximum der Menge $A$.  Das ist eine Widerspruch zu der Tatsache, dass
      $A$ eine Dedekind-Menge ist.
\end{enumerate}
\renewcommand{\labelenumi}{\arabic{enumi}.}
Bisher haben wir das Produkt $A \cdot B$ nur f�r den Fall definiert, dass $A$ und $B$ beide positiv
sind.  Falls $A$ oder $B$ gleich $O$ ist, definieren wir das Produkt als $O$:
\\[0.2cm]
\hspace*{1.3cm}
$A \cdot O := O \cdot B := O$
\\[0.2cm]
Falls $A$ weder positiv noch gleich $O$ ist, sagen wir, dass $A$  \emph{negativ} ist.  In einem
solchen Fall ist $-\!A$ positiv.  Falls $A$ oder $B$ negativ ist, lautet die Definition wie folgt:
\begin{enumerate}
\item[2.] Fall: $A$ ist positiv, aber $B$ negativ.  Dann ist $-\!B$ positiv und wir k�nnen
          \\[0.2cm]
          \hspace*{1.3cm}
          $A \cdot B := -\!\bigl(A \cdot (-\!B)\bigr)$
          \\[0.2cm]
          definieren.
\item[3.] Fall: $B$ ist positiv, aber $A$ ist negativ.  Dann setzen wir
          \\[0.2cm]
          \hspace*{1.3cm}
          $A \cdot B := -\!\bigl((-\!A) \cdot B\bigr)$.
\item[4.] Fall: $A$ und $B$ sind negativ.  Wir definieren
          \\[0.2cm]
          \hspace*{1.3cm}
          $A \cdot B := (-\!A) \cdot (-\!B)$.
\end{enumerate}
Nun m�ssten wir noch nachweisen, dass f�r die so definierte Multiplikation zusammen mit der oben definierten
Addition die K�rper-Axiome gelten.  Aus Zeitgr�nden verzichten wir darauf.


\exercise
Es sei $r \in \mathbb{Q}$ mit $r > 0$.  Beweisen Sie, dass es eine nat�rliche Zahl $n$ gibt, so dass
$r < n$ ist. \eox
\vspace*{0.3cm}

\noindent
In der nun folgenden Definition stellen wir die Eigenschaften der reellen Zahlen noch einmal in
axiomatischer Form zusammen.
\pagebreak

\begin{Definition}[Angeordneter K�rper]
  Ein \emph{angeordneter K�rper} ist ein 6-Tupel 
  \\[0.2cm]
  \hspace*{1.3cm}
  $\langle \mathbb{K}, 0, 1, +, \cdot, \leq \rangle$
  \\[0.2cm]
  f�r das Folgendes gilt:
  \begin{enumerate}
  \item $\mathbb{K}$ ist eine Menge, 
  \item $0 \in \mathbb{K}$, $1 \in \mathbb{K}$ und $0 \not= 1$,
  \item $+: \mathbb{K} \times \mathbb{K} \rightarrow \mathbb{K}$

        ist eine bin�re Operation auf $\mathbb{K}$, so dass $\langle \mathbb{K}, 0, + \rangle$
        eine kommutatve Gruppe ist, f�r alle $x,y,z \in \mathbb{K}$ gilt also:
        \begin{enumerate}
        \item $(x + y) + z = x + (y + z)$,
        \item $x + y = y + x$,
        \item $0 + x = x$,
        \item $\exists y \in \mathbb{K}: x + y = 0$.

              Dasjenige $y \in \mathbb{K}$, f�r welches $x + y = 0$ gilt, ist dann eindeutig bestimmt und wird mit
              $-x$ bezeichnet.
        \end{enumerate}
  \item $\cdot: \mathbb{K} \times \mathbb{K} \rightarrow \mathbb{K}$

        ist eine bin�re Operation auf $\mathbb{K}$, so dass $\langle \mathbb{K}\backslash\{0\}, 1, \cdot \rangle$
        eine kommutative Gruppe ist, f�r alle $x,y,z \in \mathbb{K} \backslash\{0\}$ gilt also:
        \begin{enumerate}
        \item $(x \cdot y) \cdot z = x \cdot (y \cdot z)$,
        \item $x \cdot y = y \cdot x$,
        \item $1 \cdot x = x$,
        \item $\exists y \in \mathbb{K}: x \cdot y = 1$.

              Das eindeutig bestimmte $y \in \mathbb{K}$, f�r das $x \cdot y = 1$ gilt, wird mit $\frac{1}{x}$ bezeichnet.
        \end{enumerate}
  \item Die Operationen $+$ und $\cdot$ gen�gen dem Distributiv-Gesetz
        \\[0.2cm]
        \hspace*{1.3cm}
        $x \cdot (y + z) = x \cdot y + x \cdot z$.
  \item $\leq: \mathbb{K} \times \mathbb{K} \rightarrow \mathbb{B}$
    
        ist eine bin�re Relation auf $\mathbb{K}$, so dass das Paar $\langle \mathbb{K}, \leq
       \rangle$ eine lineare Ordnung ist.  F�r alle
        $x,y,z \in \mathbb{K}$ gilt also
        \begin{enumerate}
        \item $x \leq x$,
        \item $x \leq y \wedge y \leq x \rightarrow x = y$,
        \item $x \leq y \wedge y \leq z \rightarrow x \leq z$,
        \item $x \leq y \vee y \leq x$.
        \end{enumerate}
  \item Definieren wir die bin�re Relation $<: \mathbb{K} \times \mathbb{K} \rightarrow \mathbb{B}$
        f�r alle $x,y \in \mathbb{K}$ durch die Festlegung
        \\[0.2cm]
        \hspace*{1.3cm}
        $x < y \stackrel{\mathrm{def}}{\Longleftrightarrow} x \leq y \wedge x \not= y$
        \\[0.2cm]
        so sind die Operationen $+$ und $\cdot$ mit der Relation $<$ in folgender Weise vertr�glich:
        \begin{enumerate}
        \item $0 < x \wedge 0 < y \rightarrow 0 < x + y$,
        \item $0 < x \wedge 0 < y \rightarrow 0 < x \cdot y$,
        \end{enumerate}
\end{enumerate}
Falls die Ordnung $\langle \mathbb{K}, \leq \rangle$ zus�tzlich eine vollst�ndige Ordnung ist, wenn
also zu jeder nicht-leeren nach unten beschr�nkten Menge $M \subseteq \mathbb{K}$ ein Infimum
existiert, dann ist die Struktur  $\langle \mathbb{K}, 0, 1, +, \cdot, \leq \rangle$ ein vollst�ndig
angeordneter K�rper.  \eod
\end{Definition}


\examples
\begin{enumerate}
\item Die Struktur
      \\[0.2cm]
      \hspace*{1.3cm}
      $\mathcal{Q} := \langle \mathbb{Q}, 0, 1, +, \cdot, \leq \rangle$ 
      \\[0.2cm]
      ist ein angeordneter K�rper,  aber wir hatten fr�her bereits gesehen, dass die Ordnung $\langle \mathbb{Q}, \leq \rangle$
      keine vollst�ndige Ordnung ist und daher ist $\mathcal{Q}$ kein vollst�ndig angeordneter K�rper.
\item Die Struktur 
      \\[0.2cm]
      \hspace*{1.3cm}
      $\mathcal{R} := \langle \mathbb{R}, 0, 1, +, \cdot, \leq \rangle$ 
      \\[0.2cm]
      ist ein vollst�ndig angeordneter K�rper.  Es l�sst sich zeigen, dass diese Struktur bis auf
      Isomorphie der einzige vollst�ndig angeordnete K�rper ist.  Wenn also
      \\[0.2cm]
      \hspace*{1.3cm}
      $\mathcal{K} := \langle \mathbb{K}, \widehat{0}, \widehat{1}, \widehat{+}, \hat{\cdot}, \widehat{\leq} \rangle$ 
      \\[0.2cm]
      ein weiterer vollst�ndig angeordneter K�rper ist, dann gibt es eine bijektive Funktion
      \\[0.2cm]
      \hspace*{1.3cm}
      $f: \mathbb{R} \rightarrow \mathbb{K}$
      \\[0.2cm]
      die mit den Operationen $+$, $\cdot$ und $\leq$ vertr�glich ist, es gilt dann also
      \begin{enumerate}
      \item $f(x + y) = f(x) \;\widehat{+}\; f(y)$,
      \item $f(x \cdot y) = f(x) \;\hat{\cdot}\; f(y)$,
      \item $x \leq y  \leftrightarrow f(x) \;\widehat{\leq}\; f(y)$.
      \end{enumerate}
      Damit stimmen die Strukturen $\mathcal{R}$ und $\mathcal{K}$ dann bis auf die Benennung der
      Elemente �berein. \eox
\end{enumerate}

\noindent
\textbf{Literatur-Hinweise} \\
In dem Buch ``\emph{Grundlagen der Analysis}'' von Edmund Landau \cite{landau:1930} wird die oben skizzierte
Konstruktion der reellen Zahlen im Detail beschrieben.   Auch das Buch ``\emph{Principles of
  Mathematical Analysis}'' von Walter Rudin \cite{rudin:1976} diskutiert die Konstruktion der reellen
Zahlen mit Hilfe von Dedekind-Mengen ausf�hrlich. 





\section{Geschichte}
Die Konstruktion der reellen
Zahlen mit Hilfe von Schnitten geht auf Richard Dedekind zur�ck, der die nach ihm benannten Schnitte in dem Buch
\href{http://books.google.de/books?id=n-43AAAAMAAJ&printsec=frontcover&source=gbs_ge_summary_r&cad=0#v=onepage&q&f=false}{Stetigkeit und irrationale Zahlen} 
\cite{dedekind:1872}, das im Jahre 1872 erschienen ist, beschrieben hat.  Damit war erstmals eine
formale Definition des Begriffs der reellen Zahlen gefunden worden.  Diese Definition war eine der wichtigsten
Fortschritte im Bereich der mathematischen Grundlagenforschung des 19ten Jahrhunderts, denn sie
erm�glichte es, die Analysis auf ein solides Fundament zu stellen.

%%% Local Variables: 
%%% mode: latex
%%% TeX-master: "analysis"
%%% End: 


\chapter{Folgen und Reihen}
Die Begriffe \emph{Folgen} und \emph{Reihen} sowie der Begriff des \emph{Grenzwerts} bilden die
Grundlage, auf der die Analysis aufgebaut ist.  Da Reihen nichts anderes sind als
spezielle Folgen, beginnen wir unsere Diskussion mit den Folgen.

\section{Folgen}
\begin{Definition}[Folge]
  Eine Funktion $f: \mathbb{N} \rightarrow \mathbb{R}$ bezeichnen wir als eine  \emph{reellwertige Folge}. 
  Eine Funktion $f: \mathbb{N} \rightarrow \mathbb{C}$ bezeichnen wir als eine  \emph{komplexwertige
    Folge}. \eod
\end{Definition}

\noindent
Ist die Funktion $f$ ein Folge, so schreiben wir dies k�rzer als 
$\bigl(f(n)\bigr)_{n\in\mathbb{N}}$ oder $\bigl(f_n\bigr)_{n\in\mathbb{N}}$ oder noch k�rzer als $\bigl(f_n\bigr)_n$.
\vspace*{0.3cm}

\noindent
\textbf{Beispiele}:
\begin{enumerate}
\item Die Funktion $a:\mathbb{N} \rightarrow \mathbb{R}$, die durch  
      $a(n) = \bruch{1}{n}$ definiert ist, schreiben wir als die Folge
      $\Bigl(\bruch{1}{n}\Bigr)_{n\in\mathbb{N}}$.
\item Die Funktion $a:\mathbb{N} \rightarrow \mathbb{R}$, die durch  
      $a(n) = (-1)^n$ definiert ist, schreiben wir als die Folge
      $\bigl((-1)^n\bigr)_{n\in\mathbb{N}}$.
\item Die Funktion $a:\mathbb{N} \rightarrow \mathbb{R}$, die durch  
      $a(n) = n$ definiert ist, schreiben wir als die Folge
      $\bigl(n\bigr)_{n\in\mathbb{N}}$. 
\end{enumerate}
Folgen k�nnen auch induktiv definiert werden.  Um die Gleichung $x = \cos(x)$ zu l�sen,
k�nnen wir eine Folge $(x_n)_{n\in\mathbb{N}}$ induktiv wie folgt definieren:
\begin{enumerate}
\item Induktions-Anfang: $n = 1$.  Wir setzen
      \\[0.2cm]
      \hspace*{1.3cm}
      $x_1 := 0$.
\item Induktions-Schritt: $n \mapsto n+1$.  Nach Induktions-Voraussetzung ist $x_n$ bereits
      definiert.  Wir definieren $x_{n+1}$ als
      \\[0.2cm]
      \hspace*{1.3cm}
      $x_{n+1} := \cos(x_n)$. \eod
\end{enumerate}
Wir k�nnen die ersten $40$ Glieder dieser Folge mit dem in Abbildung
\ref{fig:solve.stlx} gezeigten \href{http://randoom.org/Software/SetlX}{\textsc{SetlX}}-Programm
berechnen. Wir erhalten dann  
die in der  Tabelle \ref{tab:x-cos-x} auf Seite \pageref{tab:x-cos-x} 
gezeigten Ergebnisse.  Bei n�herer Betrachtung der Ergebnisse stellen wir fest,
dass die Folge $\folge{x_n}$ in einem gewissen Sinne gegen einen festen
\href{http://de.wikipedia.org/wiki/Grenzwert_(Folge)}{\emph{Grenzwert}} strebt.   
Diese Beobachtung wollen wir in der folgenden Definition
pr�zisieren.  Vorab 
bezeichnen wir  die Menge der positiven reellen Zahlen mit $\mathbb{R}_+$, es gilt also
\[ \mathbb{R}_+ = \bigl\{ x \in \mathbb{R} \mid x > 0 \bigl\}. \]


\begin{figure}[!ht]
  \centering
\begin{Verbatim}[ frame         = lines, 
                  framesep      = 0.3cm, 
                  labelposition = bottomline,
                  numbers       = left,
                  numbersep     = -0.2cm,
                  xleftmargin   = 1.3cm,
                  xrightmargin  = 1.3cm,
                ]
    solve := procedure(k) {
        x    := [];  // x[n+1] stores x_{n}
        x[1] := 0.0;
        for (n in [1 .. k]) {
            x[n+1] := cos(x[n]);
            print("x_{$n$} = $x[n+1]$");
        }
    };
\end{Verbatim}
\vspace*{-0.3cm}
  \caption{Berechnung der durch  $x_0 = 0$ und $x_{n+1} = \cos(x_n)$ definierten Folge.}
  \label{fig:solve.stlx}
\end{figure} %\$

\begin{table}[!h]
  \centering
\framebox{
  \begin{tabular}{|l|c|l|c|l|c|l|c|}
\hline
   $n$ & $x_n$ & $n$ & $x_n$ & $n$ & $x_n$ & $n$ & $x_n$ \\
\hline
\hline
0 & 0.000000 & 10 & 0.731404 & 20 & 0.738938 & 30 & 0.739082 \\
\hline
1 & 1.000000 & 11 & 0.744237 & 21 & 0.739184 & 31 & 0.739087 \\
\hline
2 & 0.540302 & 12 & 0.735605 & 22 & 0.739018 & 32 & 0.739084 \\
\hline
3 & 0.857553 & 13 & 0.741425 & 23 & 0.739130 & 33 & 0.739086 \\
\hline
4 & 0.654290 & 14 & 0.737507 & 24 & 0.739055 & 34 & 0.739085 \\
\hline
5 & 0.793480 & 15 & 0.740147 & 25 & 0.739106 & 35 & 0.739086 \\
\hline
6 & 0.701369 & 16 & 0.738369 & 26 & 0.739071 & 36 & 0.739085 \\
\hline
7 & 0.763960 & 17 & 0.739567 & 27 & 0.739094 & 37 & 0.739085 \\
\hline
8 & 0.722102 & 18 & 0.738760 & 28 & 0.739079 & 38 & 0.739085 \\
\hline
9 & 0.750418 & 19 & 0.739304 & 29 & 0.739089 & 39 & 0.739085 \\
\hline
  \end{tabular}}
  \caption{Die ersten 40 Glieder der durch $x_0 = 0$ und $x_{n+1} = \cos(x_n)$ definierten
    Folge.}
  \label{tab:x-cos-x}
\end{table}

\begin{Definition}[Grenzwert]
Eine Folge $\folge{a_n}$ \emph{konvergiert} gegen den \emph{Grenzwert} $g$, falls gilt:
\\[0.2cm]
\hspace*{1.3cm}
$\forall \varepsilon \in\mathbb{R}_+: \exists K \in \mathbb{R} : \forall n \in \mathbb{N} : n \geq K \rightarrow |a_n - g| < \varepsilon$. 
\\[0.2cm]
In diesem Fall schreiben wir
\\[0.2cm]
\hspace*{1.3cm}
$\lim\limits_{n\rightarrow\infty} a_n = g$.  \eod
\end{Definition}
Anschaulich besagt diese Definition, dass fast alle Glieder $a_n$ der Folge $\folge{a_n}$
einen beliebig kleinen Abstand zu dem Grenzwert $g$ haben.
F�r die oben induktiv definierte Folge $x_n$ k�nnen wir den Nachweis der Konvergenz erst
in einem sp�teren Kapitel antreten.  Wir betrachten statt dessen ein einfacheres Beispiel
und beweisen, dass 
\\[0.2cm]
\hspace*{1.3cm}
$\lim\limits_{n\rightarrow\infty} \bruch{1}{n} = 0$
\\[0.2cm]
gilt. 
\vspace*{0.2cm}

\noindent
\textbf{Beweis}:  F�r jedes  $\varepsilon > 0$ m�ssen wir eine Zahl $K$ angeben, so
dass f�r alle nat�rlichen Zahlen $n$, die gr��er-gleich $K$ sind, die Absch�tzung
\\[0.2cm]
\hspace*{1.3cm}
$ \left| \bruch{1}{n} - 0 \right| < \varepsilon $
\\[0.2cm]
gilt.  Wir definieren $K := \bruch{1}{\varepsilon} + 1$.  Damit ist $K$ wohldefiniert,
denn da $\varepsilon$ positiv ist, gilt sicher auch $\varepsilon \not= 0$.  Nun benutzen
wir die Voraussetzung $n \geq K$ f�r $K = \bruch{1}{\varepsilon} + 1$:
\\[0.2cm]
\hspace*{1.3cm}
$
\begin{array}{cll}
            & n \geq \bruch{1}{\varepsilon} + 1      \\[0.3cm] 
\Rightarrow & n > \bruch{1}{\varepsilon} & \mid \cdot \;\varepsilon \\[0.3cm]
\Rightarrow & n \cdot \varepsilon > 1       & \mid \cdot \;\bruch{1}{n} \\[0.3cm]
\Rightarrow & \varepsilon > \bruch{1}{n} & 
\end{array}
$
\\[0.2cm]
Da andererseits  $0 < \bruch{1}{n}$ gilt, haben wir insgesamt f�r alle $n > K$
\\[0.2cm]
\hspace*{1.3cm}
$
\begin{array}{cl}
            & 0 < \bruch{1}{n} < \varepsilon              \\[0.3cm]
\Rightarrow & \left|\bruch{1}{n}\right| < \varepsilon     \\[0.3cm]
\Rightarrow & \left|\bruch{1}{n}-0\right| < \varepsilon
\end{array}
$
\\[0.2cm]
gezeigt und damit ist der Beweis abgeschlossen. \hspace*{\fill} $\Box$
\vspace*{0.3cm}

\exercise
\renewcommand{\labelenumi}{(\alph{enumi})}
\begin{enumerate}
\item Beweisen Sie unter R�ckgriff auf die Definition des Grenzwert-Begriffs, dass
\\[0.2cm]
\hspace*{1.3cm}
$\lim\limits_{n\rightarrow\infty} \bruch{1}{2^n} = 0$ 
\\[0.2cm] gilt.
\item Beweisen Sie unter R�ckgriff auf die Definition des Grenzwert-Begriffs, dass
\\[0.2cm]
\hspace*{1.3cm}
$\lim\limits_{n\rightarrow\infty} \bruch{1}{\sqrt{n}} = 0$ 
\\[0.2cm] 
gilt.  \eox
\end{enumerate} 
\renewcommand{\labelenumi}{\arabic{enumi}.}
\vspace*{0.3cm}


\noindent
Wir formulieren und beweisen einige unmittelbare Folgerungen aus der obigen Definition des Grenzwerts.
\begin{Satz}[Eindeutigkeit des Grenzwerts]
Konvergiert die Folge $\folge{a_n}$ sowohl gegen den Grenzwert $g_1$ als auch gegen den
Grenzwert $g_2$, so gilt $g_1 = g_2$.
\end{Satz}
\textbf{Beweis}:  Wir f�hren den Beweis indirekt und nehmen an, dass $g_1 \not= g_2$ ist.
Dann definieren wir $\varepsilon = \bruch{1}{2}\cdot|g_2 - g_1|$ und aus der Annahme $g_1 \not= g_2$ folgt
$\varepsilon > 0$.  Aus der Voraussetzung, dass $\folge{a_n}$ gegen $g_1$ konvergiert
folgt, dass es ein $K_1$ gibt, so dass gilt:
\\[0.2cm]
\hspace*{1.3cm}
$ \forall n \in \mathbb{N}: n \geq K_1 \rightarrow | a_n - g_1 | < \varepsilon $
\\[0.2cm]
Analog folgt  aus der Voraussetzung, dass $\folge{a_n}$ gegen $g_2$ konvergiert,
dass es ein $K_2$ gibt, so dass gilt:
\\[0.2cm]
\hspace*{1.3cm}
$ \forall n \in \mathbb{N}: n \geq K_2 \rightarrow | a_n - g_2 | < \varepsilon $
\\[0.2cm]
Wir setzen $K := \max(K_1, K_2)$.  Alle $n\in\mathbb{N}$, die gr��er-gleich $K$ sind,
sind dann sowohl gr��er-gleich $K_1$ als auch gr��er-gleich $K_2$. Unter Benutzung der 
\emph{Dreiecksungleichung}\footnote{
Sind $a, b\in \mathbb{R}$, so gilt $|a+b| \leq |a| + |b|$.  Diese Ungleichung tr�gt den
Namen \emph{Dreiecksungleichung.}}
erhalten wir f�r alle $n \geq K$ die folgende Kette von Ungleichungen:
\\[0.2cm]
\hspace*{1.3cm}
$
\begin{array}{lcll}  
   2 \cdot \varepsilon & = & |g_2 - g_1| \\
                   & = & |(g_2 - a_n) + (a_n - g_1)| \\
                   & \leq & |g_2 - a_n| + |a_n - g_1| 
                          & \mbox{(Dreiecksungleichung)} \\
                   & < & \varepsilon + \varepsilon \\
                   & = & 2 \cdot \varepsilon \\
\end{array}
$
\\[0.2cm]
Aus dieser Ungleichungs-Kette w�rde aber $2\cdot \varepsilon < 2\cdot \varepsilon$ folgen und dass
ist ein Widerspruch.  Somit ist die Annahme $g_1 \not= g_2$ falsch und es muss $g_1 = g_2$
gelten.  
\hspace*{\fill} $\Box$
\vspace*{0.3cm}

\noindent
\textbf{Bemerkung}:  Die Schreibweise $\lim\limits_{n\rightarrow\infty} a_n = g$
wird durch den letzten Satz im Nachhinein gerechtfertigt.
\vspace*{0.3cm}

\exercise
Zeigen Sie, dass die Folge $\folge{(-1)^n}$ nicht konvergent ist. \eox
\vspace*{0.3cm}

\noindent 
\textbf{L�sung}: Wir f�hren den Beweis indirekt und nehmen an, dass die Folge
$\folge{(-1)^n}$ konvergiert.  Bezeichnen wir diesen Grenzwert mit $s$, so gilt also
\\[0.2cm]
\hspace*{1.3cm}
$ \forall \varepsilon \in \mathbb{R}_+: \exists K \in \mathbb{R} : \forall n \in \mathbb{N} : n \geq K \rightarrow \bigl|(-1)^n - s\bigr| < \varepsilon $
\\[0.2cm]
Daher gibt es f�r  $\varepsilon = 1$ eine Zahl $K$, so dass 
\\[0.2cm]
\hspace*{1.3cm}
$ \forall n \in \mathbb{N} : n \geq K \rightarrow \bigl|(-1)^n - s\bigr| < 1 $
\\[0.2cm]
gilt.  Da aus $n \geq K$ sicher auch $2 \cdot n \geq K$ und $2\cdot n+1 \geq K$ folgt, h�tten wir dann
f�r $n \geq K$ die beiden folgenden Ungleichungen:
\\[0.2cm]
\hspace*{1.3cm}
$
  \bigl| (-1)^{2\cdot n} - s\bigr| < 1 \quad \mbox{und} \quad
  \bigl| (-1)^{2\cdot n+1} - s\bigr| < 1 
$
\\[0.2cm]
Wegen $(-1)^{2\cdot n} = 1$ und $(-1)^{2\cdot n + 1} = -1$ haben wir also 
\begin{equation}
  \label{eq:ineq0}
  \bigl|  1 - s\bigr| < 1 \quad \mbox{und} \quad
  \bigl| -1 - s\bigr| < 1.
\end{equation}
Wegen $-1 - s = (-1) \cdot (1 + s)$ und $|a \cdot b| = |a|\cdot|b|$ k�nnen wir die letzte Ungleichung
noch vereinfachen zu 
\begin{equation}
  \label{eq:ineq0a}
  \bigl| 1 + s\bigr| < 1.
\end{equation}
Aus den beiden Ungleichungen $|1-s|<1$ und $|1+s|<1$  erhalten wir nun einen Widerspruch:
\\[0.2cm]
\hspace*{1.3cm}
$
\begin{array}{lcll}
  2 & = & \bigl| 1 + 1 \bigr| \\[0.2cm]
    & = & \bigl| (1 - s) + (s + 1) \bigr| \\[0.2cm]
    & \leq & \bigl|1 - s\bigr| + \bigl|1+s \bigr| & \quad \mbox{(Dreiecksungleichung)} \\[0.2cm]
    & <    & 1 + 1 & \quad \mbox{wegen der Ungleichungen (\ref{eq:ineq0}) und (\ref{eq:ineq0a})} \\[0.2cm]
    & =    & 2 & 
\end{array}
$
\\[0.2cm]
Fassen wir diese Ungleichungs-Kette zusammen, so haben die (offensichtlich falsche)
Ungleichung $2<2$ abgeleitet.
Damit haben wir aus der Annahme, dass die Folge gegen den Grenzwert $s$ konvergiert, einen
Widerspruch hergeleitet. \hspace*{\fill} $\Box$

\begin{Definition}[beschr�nkte Folgen]
Eine Folge $\folge{a_n}$ ist \emph{beschr�nkt}, falls es eine \emph{Schranke} $S$ gibt, so dass 
\\[0.2cm]
\hspace*{1.3cm}
$ \forall n \in \mathbb{N}: \bigl|a_n\bigr| \leq S$
\\[0.2cm]
gilt.  \eod
\end{Definition}
Die Folge $\folge{(-1)^n}$ ist durch die Schranke $S=1$ beschr�nkt, denn offenbar gilt
\\[0.2cm]
\hspace*{1.3cm}
$ \bigl|(-1)^n\bigr| = 1 \leq 1, $
\\[0.2cm]
aber die Folge $\folge{n}$ ist nicht beschr�nkt, denn sonst g�be es eine Zahl $S$, so dass
f�r alle nat�rlichen Zahlen $n$ die Ungleichung $n \leq S$ gilt.  Da es beliebig gro�e
nat�rliche Zahlen gibt, kann dies nicht sein.


\begin{Satz}[Beschr�nktheit konvergenter Folgen]
Jede konvergente Folge  ist beschr�nkt.
\end{Satz}
\textbf{Beweis}: Es sei  $\folge{a_n}$ eine konvergente Folge und es gelte
\\[0.2cm]
\hspace*{1.3cm}
$ \lim\limits_{n\rightarrow\infty} a_n = g. $
\\[0.2cm]
Dann gibt es f�r beliebige $\varepsilon > 0$ ein $K$, so dass gilt
\\[0.2cm]
\hspace*{1.3cm}
$ \forall n \in \mathbb{N}: n \geq K \rightarrow \bigl| a_n - g \bigr| < \varepsilon. $
\\[0.2cm]
Wir k�nnen also f�r $\varepsilon = 1$ ein $K$ finden, so dass
\\[0.2cm]
\hspace*{1.3cm}
$ \forall n \in \mathbb{N}: n \geq K \rightarrow \bigl| a_n - g \bigr| < 1 $
\\[0.2cm]
gilt.  Wir k�nnen ohne Einschr�nkung der Allgemeinheit davon ausgehen, dass
$K$ eine nat�rliche Zahl ist, denn wenn $K$ keine nat�rliche Zahl ist, k�nnen wir $K$
einfach durch die erste nat�rliche Zahl ersetzen, die gr��er als $K$ ist.
Dann definieren wir 
\\[0.2cm]
\hspace*{1.3cm}
$ S := \max\bigl\{ |a_0|, |a_1|, \cdots, |a_K|, 1 + |g| \bigr\}.  $
\\[0.2cm]
Wir behaupten, dass $S$ eine Schranke f�r die Folge $\folge{a_n}$ ist, wir zeigen also,
dass f�r alle $n \in \mathbb{N}$ gilt:
\\[0.2cm]
\hspace*{1.3cm}
$ |a_n| \leq S $
\\[0.2cm] 
Um diese Ungleichung nachzuweisen, f�hren wir eine Fall-Unterscheidung durch:
\begin{enumerate}
\item Fall: $n \leq K$.  Dann gilt offenbar 
      \\[0.2cm]
      \hspace*{1.3cm}      
      $|a_n| \in \bigl\{ |a_0|, |a_1|, \cdots, |a_K|, 1 + |g| \bigr\}$.
      \\[0.2cm]
      und daraus folgt sofort
      \\[0.2cm]
      \hspace*{1.3cm}      
      $|a_n| \leq \max\bigl\{ |a_0|, |a_1|, \cdots, |a_K|, 1 + |g| \bigr\} = S$.
\item Fall: $n > K$.  Dann haben wir
      \\[0.2cm]
      \hspace*{1.3cm}
$
      \begin{array}[b]{lcll}        
         |a_n| & =    & | a_n - g + g | \\[0.2cm]
               & \leq & | a_n - g | + |g| 
                      & \mbox{(Dreiecksungleichung)} \\[0.3cm]
               & <    & 1 + |g|           & \mbox{wegen $n>K$} \\[0.2cm]
               & \leq & S. \\[0.2cm]
      \end{array} \hspace*{\fill} \Box
      $
\end{enumerate}
Aus den letzten beiden S�tzen folgt nun sofort, dass die Folge $\folge{n}$ nicht
konvergiert, denn diese Folge ist noch nicht einmal beschr�nkt.


\begin{Satz}[Summe konvergenter Folgen]
Sind $\folge{a_n}$ und $\folge{b_n}$ zwei Folgen, so dass
\\[0.2cm]
\hspace*{1.3cm}
$ \lim\limits_{n\rightarrow\infty} a_n = a \quad \wedge \quad \lim\limits_{n\rightarrow\infty} b_n = b $
\\[0.2cm]
gilt, dann konvergiert die Folge $\folge{a_n + b_n}$ gegen den Grenzwert $a+b$, in
Zeichen:
\\[0.2cm]
\hspace*{1.3cm}
$ \lim\limits_{n\rightarrow\infty} \bigl(a_n + b_n\bigr) = 
   \Bigl(\lim\limits_{n\rightarrow\infty} a_n\Bigr) +\Bigl(\lim\limits_{n\rightarrow\infty} b_n\Bigr).
$
\end{Satz}

\proof
Es sei $\varepsilon > 0$ \underline{fest} vorgegeben.  Wir suchen ein $K$, so dass
\\[0.2cm]
\hspace*{1.3cm}
$ \forall n \in \mathbb{N} : n \geq K \rightarrow \bigl| \bigl(a_n + b_n\bigr) - (a + b)\bigr| < \varepsilon $
\\[0.2cm]
gilt.  Nach Voraussetzung gibt es f�r \underline{beliebi}g\underline{e} $\varepsilon' > 0$ ein $K_1$ und ein $K_2$, so dass
\\[0.2cm]
\hspace*{1.3cm}
$  \forall n \in \mathbb{N} : n \geq K_1 \rightarrow \bigl| a_n - a \bigr| < \varepsilon' 
   \quad \mbox{und} \quad
   \forall n \in \mathbb{N} : n \geq K_2 \rightarrow \bigl| b_n - b \bigr| < \varepsilon' 
$
\\[0.2cm]
gilt.  Wir setzen nun $\varepsilon' := \bruch{1}{2} \cdot \varepsilon$.  Dann gibt es also $K_1$ und
$K_2$, so dass
\\[0.2cm]
\hspace*{1.3cm}
$ 
   \forall n \in \mathbb{N} : n \geq K_1 \rightarrow \bigl| a_n - a \bigr| < \bruch{1}{2}\cdot\varepsilon 
   \quad \mbox{und} \quad
   \forall n \in \mathbb{N} : n \geq K_2 \rightarrow \bigl| b_n - b \bigr| < \bruch{1}{2}\cdot\varepsilon 
$
\\[0.2cm]
gilt.  Wir definieren $K := \max(K_1,K_2)$.  Damit gilt dann f�r alle $n \geq K$:
\\[0.2cm]
\hspace*{1.3cm}
$
\begin{array}{lcll}
  \bigl| \bigl(a_n + b_n\bigr) - (a + b) \bigr|  
 & = & \bigl| \bigl(a_n + b_n\bigr) - (a + b) \bigr| \\[0.3cm]
 & = & \bigl| \bigl(a_n - a\bigr) +  \bigl(b_n - b\bigr) \bigr| \\[0.3cm]
 & \leq & \bigl| \bigl(a_n - a\bigr) \bigr| +  \bigl|\bigl(b_n - b\bigr) \bigr| 
        & \mbox{(Dreiecksungleichung)} \\[0.3cm]
 & < & \bruch{1}{2}\cdot \varepsilon + \bruch{1}{2} \cdot \varepsilon \\[0.3cm]
 & = &  \varepsilon. 
\end{array}
$
\\[0.2cm]
Damit ist die Behauptung gezeigt. \hspace*{\fill} $\Box$
\vspace*{0.3cm}

\exercise
Zeigen Sie:
Sind $\folge{a_n}$ und $\folge{b_n}$ zwei Folgen, so dass
\\[0.2cm]
\hspace*{1.3cm}
$ \lim\limits_{n\rightarrow\infty} a_n = a \quad \wedge \quad \lim\limits_{n\rightarrow\infty} b_n = b $
\\[0.2cm]
gilt, dann konvergiert die Folge $\folge{a_n - b_n}$ gegen den Grenzwert $a-b$, in
Zeichen:
\\[0.2cm]
\hspace*{1.3cm}
$ \lim\limits_{n\rightarrow\infty} \bigl(a_n - b_n\bigr) = 
   \Bigl(\lim\limits_{n\rightarrow\infty} a_n\Bigr) -\Bigl(\lim\limits_{n\rightarrow\infty} b_n\Bigr).
$
\eox  

\begin{Satz}[Produkt konvergenter Folgen]
Sind $\folge{a_n}$ und $\folge{b_n}$ zwei Folgen, so dass
\\[0.2cm]
\hspace*{1.3cm}
$ \lim\limits_{n\rightarrow\infty} a_n = a \quad \wedge \quad \lim\limits_{n\rightarrow\infty} b_n = b $
\\[0.2cm]
gilt, dann konvergiert die Folge $\folge{a_n \cdot b_n}$ gegen den Grenzwert $a\cdot b$, in
Zeichen:
\\[0.2cm]
\hspace*{1.3cm}
$ \lim\limits_{n\rightarrow\infty} \bigl(a_n \cdot b_n\bigr) = 
   \Bigl(\lim\limits_{n\rightarrow\infty} a_n\Bigr) \cdot\Bigl(\lim\limits_{n\rightarrow\infty} b_n\Bigr).
$
\\[0.2cm]  
\end{Satz}
\textbf{Beweis}:  Es sei $\varepsilon > 0$ \underline{fest} vorgegeben.  Wir suchen ein $K$, so dass
\\[0.2cm]
\hspace*{1.3cm}
$ \forall n \in \mathbb{N} : n \geq K \rightarrow \bigl| \bigl(a_n \cdot b_n\bigr) - (a \cdot b)\bigr| < \varepsilon $
\\[0.2cm]
gilt.  Da die Folge $\folge{a_n}$ konvergent ist, ist diese Folge auch beschr�nkt, es gibt
also eine Zahl $S$, so dass  
\\[0.2cm]
\hspace*{1.3cm}
$ |a_n| \leq S \quad \mbox{f�r alle $n\in\mathbb{N}$}$
\\[0.2cm]
gilt.
Nach Voraussetzung gibt es f�r beliebige $\varepsilon_1 > 0$ ein
$K_1$ und f�r beliebige $\varepsilon_2 >0$ ein $K_2$, so dass
\\[0.2cm]
\hspace*{1.3cm}
$ 
   \forall n \in \mathbb{N} : n \geq K_1 \rightarrow \bigl| a_n - a \bigr| < \varepsilon_1
   \quad \mbox{und} \quad
   \forall n \in \mathbb{N} : n \geq K_2 \rightarrow \bigl| b_n - b \bigr| < \varepsilon_2
$
\\[0.2cm]
gilt.  Wir setzen nun $\varepsilon_1 := \bruch{\varepsilon}{2\cdot(|b| + 1)}$ und
$\varepsilon_2 := \bruch{\varepsilon}{2\cdot S}$.  Dann gibt es also $K_1$ und
$K_2$, so dass
\\[0.2cm]
\hspace*{1.3cm}
$ 
   \forall n \in \mathbb{N} : n \geq K_1 \rightarrow \bigl| a_n - a \bigr| < \bruch{\varepsilon}{2\cdot(|b| + 1)}
   \quad \mbox{und} \quad
   \forall n \in \mathbb{N} : n \geq K_2 \rightarrow \bigl| b_n - b \bigr| <
   \bruch{\varepsilon}{2\cdot  S}
$
\\[0.2cm]
gilt.  Wir definieren $K := \max(K_1,K_2)$.  Damit gilt dann f�r alle $n \geq K$:
\\[0.2cm]
\hspace*{1.3cm}
$
\begin{array}{lcll}
  \bigl| a_n \cdot  b_n - a \cdot  b \bigr|  
 & =    & \bigl| \bigl(a_n \cdot  b_n - a_n\cdot b \bigr) + \bigl(a_n\cdot b - a\cdot b\bigr) \bigr| \\[0.3cm]
 & \leq & \bigl| \bigl(a_n \cdot  b_n - a_n\cdot b \bigr) \bigr| + \bigl| \bigl(a_n\cdot b - a\cdot b\bigr) \bigr| 
        & \mbox{(Dreiecksungleichung)}  \\[0.3cm]
 & =    & \bigl|a_n\bigr| \cdot  \bigl| b_n - b \bigr| + \bigl| a_n - a \bigr| \cdot  |b| \\[0.3cm]
 & \leq & S \cdot  \bigl| b_n - b \bigr| + \bigl| a_n - a \bigr| \cdot  (|b| + 1) \\[0.3cm]
 & <    & S \cdot  \bruch{\varepsilon}{2\cdot S} + \bruch{\varepsilon}{2\cdot (|b| + 1)} \cdot  (|b| + 1)  \\[0.3cm]
 & \leq & \bruch{\;\varepsilon\;}{2} + \bruch{\;\varepsilon\;}{2}  \\[0.3cm]
 & =    & \varepsilon. 
\end{array}
$
\\[0.2cm]
Damit ist die Behauptung gezeigt. \hspace*{\fill} $\Box$
\vspace*{0.3cm}

\exercise
Zeigen Sie:
{\em
Sind $\folge{a_n}$ und $\folge{b_n}$ zwei Folgen, so dass
\\[0.2cm]
\hspace*{1.3cm}
$ \lim\limits_{n\rightarrow\infty} a_n = a \quad \wedge \quad \lim\limits_{n\rightarrow\infty} b_n = b $
\\[0.2cm]
gilt und gilt $b_n \not= 0$ f�r alle $n \in \mathbb{N}$, sowie $b \not= 0$,
so konvergiert die Folge $\folge{a_n/b_n}$ gegen den Grenzwert $a/b$, in
Zeichen:}
\\[0.2cm]
\hspace*{1.3cm}
$ \lim\limits_{n\rightarrow\infty} \bruch{a_n}{b_n} = 
   \bruch{\Bigl(\lim\limits_{n\rightarrow\infty}
     a_n\Bigr)}{\Bigl(\lim\limits_{n\rightarrow\infty} b_n\Bigr)} =
   \bruch{a}{b}.
$
\eox

\noindent
\textbf{L�sung}:  Zun�chst k�nnen wir das Problem vereinfachen, wenn wir die Folge
$\folge{a_n/b_n}$ als Folge von Produkten schreiben:
\\[0.2cm]
\hspace*{1.3cm}
$ \Folge{\bruch{a_n}{b_n}} = \folge{a_n} \cdot  \Folge{\bruch{1}{b_n}} $
\\[0.2cm]
Falls wir zeigen k�nnen, dass 
\\[0.2cm]
\hspace*{1.3cm}
$ \lim\limits_{n\rightarrow\infty} \bruch{1}{b_n} = \bruch{1}{b} $
\\[0.2cm]
gilt, dann folgt die Behauptung aus dem Satz �ber das Produkt konvergenter Folgen.
Bei unserer Suche nach einem Beweis starten wir damit, dass wir die Behauptung noch einmal
hinschreiben:
\begin{equation}
  \label{eq:ab0}  
\forall \varepsilon \in \mathbb{R}_+: \exists K \in \mathbb{R}:\forall n \in \mathbb{N}: 
   n \geq K \rightarrow \left| \bruch{1}{b_n} - \bruch{1}{b} \right| < \varepsilon
\end{equation}
Wir m�ssen also f�r alle $\varepsilon>0$ ein $K$ finden, so dass f�r alle nat�rlichen
Zahlen  $n>K$ die Ungleichung
\begin{equation}
  \label{eq:ab1}
  \left| \bruch{1}{b_n} - \bruch{1}{b} \right| < \varepsilon 
\end{equation}
gilt.  Irgendwie m�ssen wir die Voraussetzung, dass die Folge $\folge{b_n}$ gegen
$b$ konvergiert, ausnutzen.  Diese Voraussetzung lautet ausgeschrieben
\begin{equation}
  \label{eq:ab2}
  \forall \varepsilon' \in \mathbb{R}_+: \exists K' \in \mathbb{R}:\forall n \in \mathbb{N}: 
   n \geq K \rightarrow \bigl| b_n - b \bigr| < \varepsilon' 
\end{equation}
Wir zeigen zun�chst eine Absch�tzung f�r die Betr�ge $|b_n|$, die wir sp�ter brauchen.
Hier hilft uns die Voraussetzung, dass $b \not= 0$ ist.  Setzen wir in Ungleichung 
(\ref{eq:ab2}) f�r $\varepsilon'$ den Wert $\bruch{1}{2} \cdot |b|$ ein, so erhalten wir eine Zahl
$K_1$, so dass f�r alle nat�rlichen Zahlen $n \geq K_1$
\\[0.2cm]
\hspace*{1.3cm}
$ \bigl| b_n - b \bigr| < \bruch{1}{2}\cdot |b| $
\\[0.2cm]
gilt.  Damit folgt:
\\[0.2cm]
\hspace*{1.3cm}
$\begin{array}{lrcl}
                & \bigl| b \bigr| & =    & \bigl| b - b_n + b_n \bigr|                   \\[0.2cm]
    \Rightarrow & \bigl| b \bigr| & \leq & \bigl| b - b_n \bigr| + \bigl| b_n \bigr|     \\[0.2cm]
    \Rightarrow & \bigl| b \bigr| & <    & \bruch{1}{2} \cdot \bigl|b\bigr| + \bigl| b_n \bigr| \\[0.2cm]
    \Rightarrow & \bruch{1}{2} \cdot \bigl|b\bigr| & < & \bigl| b_n \bigr| \\[0.3cm]
    \Rightarrow & \bruch{2}{\bigl|b\bigr|} & > & \bruch{1}{\bigl| b_n \bigr|}
  \end{array}
$
\\[0.2cm]
Damit wissen wir also, dass f�r alle $n>K_1$ die Ungleichung
\\[0.2cm]
\hspace*{1.3cm}
$ \bruch{1}{2}\cdot |b| < |b_n| $
\\[0.2cm]
gilt.  Um nun f�r ein gegebenes $\varepsilon > 0$ die Ungleichung (\ref{eq:ab1}) zu zeigen,  setzen wir
in der Voraussetzung (\ref{eq:ab2})   $\varepsilon' = \bruch{1}{2}\cdot |b|^2 \cdot  \varepsilon$ 
und erhalten ein $K_2$, so dass f�r alle $n>K_2$ die Ungleichung
\begin{equation}
  \label{eq:ab3}
  \bigl|b - b_n\bigr| < \bruch{1}{2} \cdot  |b|^2 \cdot  \varepsilon
\end{equation}
gilt.  Setzen wir $K := \max(K_1,K_2)$, so erhalten wir f�r alle $n>k$  die folgende
Ungleichungs-Kette: 
\\[0.2cm]
\hspace*{1.3cm}
$
\begin{array}{lcll}
       \left| \bruch{1}{b_n} - \bruch{1}{b} \right| 
 & = & \left| \bruch{b-b_n}{b\cdot b_n} \right| \\[0.5cm]
 & = & \bruch{1}{|b|\cdot |b_n|} \cdot  \left| b-b_n\right| \\[0.5cm]
 & < & \bruch{2}{|b|\cdot |b|} \cdot  \left| b-b_n\right| 
     & \mbox{wegen $\bruch{2}{\bigl|b\bigr|} > \bruch{1}{\bigl| b_n \bigr|}$} 
       \\[0.5cm]
 & < & \bruch{2}{|b|\cdot |b|} \cdot  \bruch{1}{2}\cdot |b|^2 \cdot  \varepsilon 
     & \mbox{wegen (\ref{eq:ab3})} \\[0.5cm]
 & = & \varepsilon \\[0.3cm]
\end{array}
$
\\[0.2cm]
Damit haben wir f�r $n \geq K$ die Ungleichung
$\left| \bruch{1}{b_n} - \bruch{1}{b} \right| < \varepsilon$
hergeleitet und der Beweis ist abgeschlossen. \hspace*{\fill} $\Box$
\vspace*{0.3cm}

\noindent
Die bisher bewiesenen S�tzen k�nnen wir benutzen um die Grenzwerte 
von Folgen zu berechnen.  Wir geben ein Beispiel:
\\[0.2cm]
\hspace*{1.3cm}
$
\begin{array}[b]{lcl}  
       \lim\limits_{n\rightarrow\infty} \bruch{n}{n+1}                                 
& = &  \lim\limits_{n\rightarrow\infty} \bruch{1}{1+\bruch{1}{n}}                                 \\[0.8cm]
& = &  \bruch{\lim\limits_{n\rightarrow\infty} 1}{\lim\limits_{n\rightarrow\infty}1 +\bruch{1}{n}} \\[0.8cm]
& = &  \bruch{1}{\lim\limits_{n\rightarrow\infty}1 + \lim\limits_{n\rightarrow\infty} \bruch{1}{n}} \\[0.8cm]
& = &  \bruch{1}{1 + 0}                                                                             \\[0.3cm]
& = &  1
\end{array}
$
\eox



\begin{Satz}
Sind $\folge{a_n}$  
und $\folge{b_n}$ zwei konvergente Folgen, so dass
\\[0.2cm]
\hspace*{1.3cm}
$ \forall n \in \mathbb{N}: a_n \leq b_n $
\\[0.2cm]
gilt, dann gilt auch
\\[0.2cm]
\hspace*{1.3cm}
$ \lim\limits_{n\rightarrow\infty} a_n \leq \lim\limits_{n\rightarrow\infty} b_n. $
\end{Satz}

\exercise
Beweisen Sie den letzten Satz.  \eox


\begin{Definition}[monoton]
Eine Folge $\folge{a_n}$ ist \emph{monoton steigend} falls 
\\[0.2cm]
\hspace*{1.3cm}
$ \forall n \in \mathbb{N}: a_n \leq a_{n+1} $
\\[0.2cm]
gilt.  Analog hei�t eine Folge \emph{monoton fallend} falls
\\[0.2cm]
\hspace*{1.3cm}
$ \forall n \in \mathbb{N}: a_n \geq a_{n+1}. $
\eod
\end{Definition}
Ein Beispiel f�r eine monoton fallende Folge ist die Folge $\Folge{\bruch{1}{n}}$, denn es
gilt
\\[0.2cm]
\hspace*{1.3cm}
$
\begin{array}{crcll}
            &  n+1          & \geq & n             & \mid \cdot  \bruch{1}{n} \\[0.3cm]
\Rightarrow & \bruch{n+1}{n} & \geq & 1             & \mid \cdot  \bruch{1}{n+1}\\[0.3cm]
\Rightarrow & \bruch{1}{n}   & \geq & \bruch{1}{n+1} \\[0.3cm]
\end{array}
$


\begin{Satz} \label{satz:monoton}
Ist die Folge $\folge{a_n}$ monoton fallend und beschr�nkt, so ist die Folge auch konvergent.
\end{Satz}
\textbf{Beweis}:  Wir definieren zun�chst die Menge $M$ als als die Menge aller unteren Schranken der Folge $\folge{a_n}$
\\[0.2cm]
\hspace*{1.3cm}      
$M := \bigl\{ x \in \mathbb{Q} \mid \forall n \in \mathbb{N}: x \leq a_n \bigr\}$.
\\[0.2cm]
Weil wir vorausgesetzt haben, dass die Folge $\folge{a_n}$ beschr�nkt ist,
ist die Menge $M$ sicher nicht leer.  Au�erdem ist die Menge $M$ nach oben beschr�nkt, eine obere
Schranke ist das Folgenglied $a_1$.  Folglich hat die Menge $M$ ein Supremum und wir k�nnen daher
\\[0.2cm]
\hspace*{1.3cm}
$s := \sup(M)$
\\[0.2cm]
definieren.  Wir zeigen, dass
\\[0.2cm]
\hspace*{1.3cm}
$\lim\limits_{n\rightarrow\infty} a_n = s$
\\[0.2cm]
gilt.  Sie also $\varepsilon > 0$ gegeben.  Da 
\\[0.2cm]
\hspace*{1.3cm}
$s + \varepsilon > s$
\\[0.2cm]
ist und $s$ als das Supremum der Menge $M$ definiert ist, k�nnen wir folgern, dass 
$s + \varepsilon \not\in M$  ist.  Nach Definition der Menge $M$ gibt es dann eine Zahl $\widehat{n} \in \mathbb{N}$,
so dass 
\\[0.2cm]
\hspace*{1.3cm}
$a_{\widehat{n}} < s + \varepsilon$ ist.
\\[0.2cm]
Da die Folge monoton fallend ist, gilt dann auch
\\[0.2cm]
\hspace*{1.3cm}
$a_n < s + \varepsilon$ \quad f�r alle $n \geq \widehat{n}$.
\\[0.2cm]
Andererseits ist $s - \frac{1}{2} \cdot \varepsilon < s$, so dass $s - \frac{1}{2} \cdot \varepsilon$ sicher ein Element der Menge $M$
ist und damit dann auch eine untere Schranke der Folge $(a_n)_{n\in\mathbb{N}}$.  Folglich gilt f�r
alle $n \in \mathbb{N}$
\\[0.2cm]
\hspace*{1.3cm}
$s - \varepsilon < s - \frac{1}{2} \cdot \varepsilon \leq a_n$.
\\[0.2cm]
Damit haben wir insgesamt
\\[0.2cm]
\hspace*{1.3cm}
$s - \varepsilon < a_n < s + \varepsilon$ \quad f�r alle $n \geq \widehat{n}$
\\[0.2cm]
und dies k�nnen wir auch als
\\[0.2cm]
\hspace*{1.3cm}
$|a_n - s| < \varepsilon$ \quad f�r alle $n \geq \widehat{n}$
\\[0.2cm]
schreiben.  Nach Definition des Grenzwerts haben wir damit die Behauptung gezeigt.
\qed


\exercise
Die Folge $\folge{a_n}$ sei monoton steigend und beschr�nkt.  Zeigen Sie, dass der Grenzwert
\\[-0.2cm]
\hspace*{1.3cm}
$\lim\limits_{n\rightarrow\infty} a_n$
\\[0.2cm]
existiert.  \eox


\exercise 
Es seien $a,b \in \mathbb{R}$ und es gelte $a < b$.  Zeigen Sie, dass dann auch
\\[0.2cm]
\hspace*{1.3cm} 
$a < \bruch{1}{2} \cdot (a + b)$ \quad und \quad 
$\bruch{1}{2} \cdot (a + b) < b$ 
\\[0.2cm]
gilt.  \eox



\begin{Definition}[Cauchy-Folge] \lb
Eine Folge $\folge{a_n}$ hei�t \emph{Cauchy-Folge} (\href{http://de.wikipedia.org/wiki/Augustin-Louis_Cauchy}{\textrm{Augustin-Louis Cauchy}}, 1789-1857), falls gilt:
\\[0.2cm]
\hspace*{1.3cm}
$ \forall \varepsilon \in \mathbb{R}_+: \exists K \in \mathbb{R}: \forall m,n \in \mathbb{N}: 
   m \geq K \wedge n \geq K \rightarrow \bigl| a_m - a_n \bigr| < \varepsilon.
$
\eod
\end{Definition}

In einer Cauchy-Folge $\folge{a_n}$ liegen also die einzelnen Folgenglieder $a_n$ mit wachsendem $n$
immer dichter zusammen.  Wir werden sehen, dass eine Folge genau dann konvergent ist, wenn die Folge
eine Cauchy-Folge ist.


\begin{Satz}
Jede konvergente Folge $\folge{a_n}$ ist eine Cauchy-Folge.  
\end{Satz}


\noindent
\textbf{Beweis}:  Es sei $a := \lim\limits_{n\rightarrow\infty} a_n$. 
Sei $\varepsilon > 0$ gegeben.  Aufgrund der Konvergenz der Folge $\folge{a_n}$ gibt es
dann ein $K$, so dass 
\\[0.2cm]
\hspace*{1.3cm}
$ \forall n \in \mathbb{N}: n \geq K \rightarrow \bigl|a_n-a\bigr| < \bruch{\varepsilon}{2} $
\\[0.2cm]
gilt.  Damit gilt f�r alle $m,n\in\mathbb{N}$ mit $m \geq K$ und $n \geq K$ die folgende Absch�tzung:
\\[0.2cm]
\hspace*{1.3cm}
$
   \begin{array}{lcl}
     |a_m - a_n| &   =  & \bigl|(a_m - a) + (a - a_n)\bigr| \\[0.2cm]
                 & \leq & \bigl|a_m - a\bigr| \;+\; \bigr|a - a_n\bigr| \\[0.2cm]
                 &   <  & \bruch{\varepsilon}{2} + \bruch{\varepsilon}{2} \\[0.2cm]
                 &   =  & \varepsilon     
   \end{array}
$
\\[0.2cm]
Damit ist gezeigt, dass $\folge{a_n}$ eine Cauchy-Folge ist. \hspace*{\fill} $\Box$

\begin{Satz}
  Jede Cauchy-Folge ist beschr�nkt.
\end{Satz}

\noindent
\textbf{Beweis}:  Wenn $\folge{a_n}$ eine Cauchy-Folge ist, dann finden wir eine Zahl
$K$, so dass f�r alle nat�rlichen Zahlen $m,n$, die gr��er-gleich $K$ sind, die Ungleichung
\\[0.2cm]
\hspace*{1.3cm}
$ \bigl| a_n - a_m | < 1 $
\\[0.2cm]
gilt.  Sei nun $h$ eine nat�rliche Zahl, die gr��er als $K$ ist.  Wir definieren
\\[0.2cm]
\hspace*{1.3cm}
$ S := \max\bigl\{ |a_1|, |a_2|, \cdots,  |a_h|, 1+ |a_h|\bigr\} $
\\[0.2cm]
und zeigen, dass $S$ eine  Schranke der Cauchy-Folge $\folge{a_n}$ ist, wir zeigen also
\\[0.2cm]
\hspace*{1.3cm}
$ \forall n \in \mathbb{N} : |a_n| \leq S. $
\\[0.2cm]
Falls $n \leq h$ ist, ist diese Ungleichung evident.  F�r alle $n>h$ haben wir die folgende
Absch�tzung: 
\\[0.2cm]
\hspace*{1.3cm}
$ 
\begin{array}{lcl}
  \bigl| a_n \bigr| & =    & \bigl| a_n - a_h + a_h \bigr| \\[0.2cm]
                    & \leq & \bigl| a_n - a_h \bigr| + \bigl| a_h \bigr| \\[0.2cm]
                    & <    & 1 + \bigl| a_h \bigr| \\[0.2cm]
                    & \leq & S. \\[0.2cm]
\end{array}
$
\\[0.2cm]
Damit ist der Beweis abgeschlossen. \hspace*{\fill} $\Box$ 

\begin{Theorem}
Jede Cauchy-Folge ist konvergent.  
\end{Theorem}
\textbf{Beweis}: Der Beweis verl�uft �hnlich wie der Nachweis, dass eine monotone und
beschr�nkte Folge konvergent ist.  Wir definieren zun�chst eine Menge $M$ wie folgt:
\\[0.2cm]
\hspace*{1.3cm}
$M := \bigl\{ x \in \mathbb{R} \mid \exists K \in \mathbb{N}: \forall n \in \mathbb{N}: n \geq K \rightarrow x \leq a_n \bigr\}$.
\\[0.2cm]
Anschaulich ist $M$ die Menge aller unteren Grenzen f�r die Mehrheit der
Folgenglieder:  Ist $x \in M$, so m�ssen von einem bestimmten Index $K$ an
alle weiteren Folgenglieder $a_n$ durch $x$ nach unten abgesch�tzt werden.
Wir nennen $M$ daher die Menge der \emph{unteren Majorit�ts-Schranken} der Folge
$(a_n)_{n\in\mathbb{N}}$, denn jedes Element aus $M$ ist eine untere Schranke f�r die Mehrheit der
Folgenglieder.
      
Die Beschr�nktheit der Cauchy-Folge impliziert, dass die Menge $M$ nicht leer ist,
denn wenn f�r alle $n \in \mathbb{N}$ die Ungleichung $|a_n| \leq S$ gilt,  dann gilt insbesondere
$-S \leq a_n$ und daraus folgt sofort $-S \in M$.  Au�erdem ist die Menge $M$ nach oben
beschr�nkt, denn die Folge $(a_n)_{n\in\mathbb{N}}$ ist durch $S$ nach oben beschr�nkt.  Als nicht-leere und nach oben
beschr�nkte Menge hat $M$ ein Supremum.  Wir definieren 
\\[0.2cm]
\hspace*{1.3cm}
$s := \sup(M)$
\\[0.2cm]
und zeigen, dass
\\[0.2cm]
\hspace*{1.3cm}
$ \lim\limits_{n\rightarrow\infty} a_n = s $
\\[0.2cm]
gilt.  Sei $\varepsilon > 0$ gegeben.  Wir suchen eine Zahl $K$, so dass f�r alle 
nat�rlichen Zahlen $n \geq K$ die Ungleichung
\\[0.2cm]
\hspace*{1.3cm}
$ \bigl|a_n - s \bigr| < \varepsilon $
\\[0.2cm]
gilt.  Wir betrachten zun�chst die Zahl $s-\bruch{\varepsilon}{2}$.  Wegen
$s-\bruch{\varepsilon}{2} < s$ und $s = \sup(M)$ folgt $s-\bruch{\varepsilon}{2} \in M$.  Damit existiert
dann nach Definition der Menge $M$ als Menge der unteren Majorit�ts-Schranken
eine Zahl $K_1$, so dass f�r alle $n\in\mathbb{N}$ mit $n\geq K_1$ die Ungleichung
\begin{equation}
  \label{eq:ineq3}
  s-\bruch{\varepsilon}{2} \leq a_n
\end{equation}
gilt.  
Da die Folge $\folge{a_n}$ eine Cauchy-Folge ist, gibt es eine Zahl $K_2$, so dass
f�r alle $m,n\in\mathbb{N}$ mit $m>K_2$ und $n>K_2$ die Ungleichung
\begin{equation}
  \label{eq:ineq4}
  \bigl| a_n - a_m \bigr| < \bruch{\varepsilon}{2}
\end{equation}
gilt.  Wir setzen nun $K = \max(K_1,K_2)$ und betrachten die Zahl
$s+\bruch{\varepsilon}{2}$, die wegen 
$s < s+\bruch{\varepsilon}{2}$ sicher kein Element von $M$ mehr ist.  Nach Definition von $M$ finden wir
dann eine nat�rliche Zahl $m$, die gr��er als $K$ ist, so dass
\begin{equation}
  \label{eq:ineq5}
  a_m < s + \bruch{\varepsilon}{2} 
\end{equation}
gilt.  F�r diese Zahl $m$ gilt sicher auch die Ungleichung (\ref{eq:ineq3}), so dass wir
insgesamt
\\[0.2cm]
\hspace*{1.3cm}
$ s - \bruch{\varepsilon}{2} \leq a_m < s + \bruch{\varepsilon}{2} $
\\[0.2cm]
haben.  Daraus folgt sofort
\begin{equation}
  \label{eq:ineq6}
 \bigl| a_m - s \bigr| \leq \bruch{\varepsilon}{2}.  
\end{equation}
Aufgrund der Ungleichung (\ref{eq:ineq4}) haben wir jetzt f�r alle nat�rlichen Zahlen
$n>K$ die folgende Kette von Ungleichungen:
\\[0.2cm]
\hspace*{1.3cm}
$
\begin{array}{lcl}
  \bigl| a_n - s \bigr| &   =  & \bigl| (a_n - a_m) + (a_m - s) \bigr| \\[0.2cm] 
                        & \leq & \bigl| (a_n - a_m) \bigr| + \bigl| (a_m - s) \bigr| \\[0.2cm] 
                        &  <   & \bruch{\varepsilon}{2} + \bruch{\varepsilon}{2}  \\[0.2cm] 
                        &  =   & \varepsilon  
\end{array}
$
\\[0.2cm]
Damit ist der Beweis abgeschlossen. \hspace*{\fill} $\Box$ 


\section{Berechnung der Quadrat-Wurzel}
Wir pr�sentieren nun eine Anwendung der bisher entwickelte Theorie und zeigen, wie die
Quadrat-Wurzel einer reellen Zahl berechnet werden kann.  Es sei eine reelle Zahl $a>0$
gegeben.  Gesucht ist eine reelle Zahl $b>0$, so dass $b^2 = a$ ist. Unsere Idee ist es,
die Zahl $b$ iterativ als L�sung einer Fixpunkt-Gleichung zu berechnen.  
Wir definieren eine Folge $b_n$ induktiv wie folgt:
\begin{enumerate}
\item[I.A.:] $n=1$.  
      \\[0.2cm]
      \hspace*{1.3cm}      
      $b_1 := \left\{ \begin{array}{ll}
                      1 & \mbox{falls}\; a \leq 1, \\
                      a & \mbox{sonst}.
               \end{array}\right.
      $
\item[I.S.:] $n \mapsto n+1$.
      \\[0.2cm]
\hspace*{1.3cm}
$ b_{n+1} := \bruch{1}{2}\cdot  \left(b_n + \bruch{a}{b_n}\right). $
\\[0.2cm]
\end{enumerate}
Um diese Definition zu verstehen, nehmen wir zun�chst an, dass der Grenzwert dieser Folge
existiert und den Wert $b \not= 0$ hat.  Dann gilt
\\[0.2cm]
\hspace*{1.3cm}
$ \begin{array}{lcl}
   b & = &  \lim\limits_{n\rightarrow\infty} b_n     \\[0.3cm]
     & = &\lim\limits_{n\rightarrow\infty} b_{n+1}   \\[0.3cm]
     & = &\lim\limits_{n\rightarrow\infty} \bruch{1}{2}\cdot  \left(b_n + \bruch{a}{b_n}\right) \\[0.3cm]
     & = & \bruch{1}{2}\cdot  \Bigl(\lim\limits_{n\rightarrow\infty} b_n + \bruch{a}{\lim\limits_{n\rightarrow\infty}b_n}\Bigr) \\[0.3cm]
     & = & \bruch{1}{2}\cdot  \left(b + \bruch{a}{b}\right) \\[0.3cm]
\end{array}
$
\\[0.2cm]
Damit ist $b$ also eine L�sung der Gleichung 
$b = \bruch{1}{2}\cdot  \left(b + \bruch{a}{b}\right)$.  
Wir formen diese Gleichung um:
\\[0.2cm]
\hspace*{1.3cm}
$
\begin{array}{lcll}
                & b     = \bruch{1}{2}\cdot  \left(b + \bruch{a}{b}\right) & \mid \cdot  2 \\[0.4cm]
\Leftrightarrow & 2 \cdot  b = b + \bruch{a}{b}                           & \mid - b  \\[0.3cm]
\Leftrightarrow & b = \bruch{a}{b}                                   & \mid \cdot  b  \\[0.3cm]
\Leftrightarrow & b^2 = a                                            & \mid \sqrt{\;\;}  \\[0.2cm]
\Leftrightarrow & b  = \sqrt{a}                                      & 
\end{array}
$
\\[0.2cm]
Falls die oben definierte Folge $\folge{b_n}$ einen Grenzwert hat, dann ist dieser
Grenzwert also die Wurzel der Zahl $a$.  Wir werden die Konvergenz der Folge nachweisen
indem wir zeigen, dass die Folge $\folge{b}$ einerseits monoton fallend und andererseits 
nach unten beschr�nkt ist.  Dazu betrachten wir zun�chst die Differenz $b_{n+1}^2 - a$:
\\[0.2cm]
\hspace*{1.3cm}
$
\begin{array}{lcl}
  b_{n+1}^2 - a & =    & \bruch{1}{4} \cdot  \left(b_n + \bruch{a}{b_n}\right)^2 - a \\[0.4cm]
                & =    & \bruch{1}{4} \cdot  \left(b_n^2 + 2\cdot a + \bruch{a^2}{b_n^2}\right) - a \\[0.4cm]
                & =    & \bruch{1}{4} \cdot  \left(b_n^2 - 2\cdot a + \bruch{a^2}{b_n^2}\right) \\[0.4cm]
                & =    & \bruch{1}{4} \cdot  \left(b_n - \bruch{a}{b_n}\right)^2 \\[0.4cm]
                & \geq & 0, 
\end{array}
$
\\[0.2cm]
denn das Quadrat einer reellen Zahl ist immer gr��er-gleich Null.
Addieren wir auf beiden Seiten der Ungleichung
\\[0.2cm]
\hspace*{1.3cm}
$b_{n+1}^2 - a \geq 0$
\\[0.2cm]
die Zahl  $a$, so haben wir
\\[0.2cm]
\hspace*{1.3cm}
 $b_{n+1}^2 \geq a$ \quad und damit auch \quad $b_{n+1} \geq \sqrt{a}$ \quad f�r alle $n \in \mathbb{N}$
\\[0.2cm]
gezeigt.  Nach unserer Definition der Folge $\folge{b_n}$ gilt diese Ungleichung auch f�r
den ersten Wert $n=1$, so dass wir also insgesamt die Ungleichung
\\[0.2cm]
\hspace*{1.3cm}
$b_n^2 \geq a$ \quad und \quad $b_n \geq \sqrt{a}$ \quad f�r alle $n \in \mathbb{N}$
\\[0.2cm]
gezeigt haben.   Daraus folgt, dass $\sqrt{a}$ eine untere Schranke der Folge $\folge{b_n}$
ist.  Dividieren wir die erste Ungleichung durch $b_{n}$, so folgt
\\[0.2cm]
\hspace*{1.3cm}
$ b_{n} \geq \bruch{a}{b_{n}}.  $
\\[0.2cm]
Die Zahl $\bruch{1}{2}\cdot \left(b_n + \bruch{a}{b_n}\right)$ ist der arithmetische
Mittelwert der Zahlen $b_n$ und $\bruch{a}{b_n}$ und muss daher zwischen diesen beiden
Zahlen liegen:
\\[0.2cm]
\hspace*{1.3cm}
$ b_{n} \geq \bruch{1}{2}\cdot \left(b_n + \bruch{a}{b_n}\right) \geq \bruch{a}{b_n}. $
\\[0.2cm]
Dieser Mittelwert ist aber gerade $b_{n+1}$, es gilt also
\\[0.2cm]
\hspace*{1.3cm}
$ b_{n} \geq b_{n+1} \geq \bruch{a}{b_n}. $
\\[0.2cm]
Dies zeigt, dass die Folge $\folge{b_n}$ monoton fallend ist und da wir oben gesehen haben, dass die
Folge durch $\sqrt{a}$ nach unten beschr�nkt ist, konvergiert die Folge.
Wir hatten oben schon gezeigt, dass der Grenzwert dieser Folge dann den Wert $\sqrt{a}$ haben muss, es
gilt also
\\[0.2cm]
\hspace*{1.3cm}
$ \lim\limits_{n\rightarrow\infty} b_n = \sqrt{a} $
\\[0.2cm]
Abbildung \ref{fig:sqrt.stlx} auf Seite \pageref{fig:sqrt.stlx}
zeigt die Definition einer Prozedur \texttt{mySqrt}()
in \textsc{SetlX}, die die ersten 9 Glieder der Folge  
berechnet und dann jeweils mit Hilfe der Funktion $\texttt{nDecimalPlaces}()$ die ersten
100 Stellen der Werte ausgibt.  

Die von diesem Programm berechnete Ausgabe ist in
Abbildung \ref{fig:sqrt-output} gezeigt.  Sie k�nnen sehen, dass die Folge sehr schnell
konvergiert.   $b_2$ stimmt auf 2 Stellen hinter dem Komma mit dem Ergebnis �berein,
bei $b_3$ sind es bereits 5 Stellen, bei $b_4$ sind es 11 Stellen, bei $b_5$ sind es 23
Stellen, bei $b_6$ sind es 47 Stellen, bei $b_7$ haben wir 96 Stellen und ab dem Folgeglied $b_8$ 
�ndern sich die ersten 100 Stellen hinter dem Komma nicht mehr.  

In modernen Mikroprozessoren wird �brigens eine verfeinerte Version des in diesem
Abschnitt beschriebenen Verfahrens eingesetzt.  Die Verfeinerung besteht im wesentlichen
darin, dass zun�chst ein guter Startwert $b_1$ in einer Tabelle nachgeschlagen wird, die
restlichen Folgeglieder werden dann in der Tat �ber die Rekursionsformel 
      $b_{n+1} = \bruch{1}{2}\cdot  \left(b_n + \bruch{a}{b_n}\right)$
berechnet.

\begin{figure}[!ht]
  \centering
\begin{Verbatim}[ frame         = lines, 
                  framesep      = 0.3cm, 
                  labelposition = bottomline,
                  numbers       = left,
                  numbersep     = -0.2cm,
                  xleftmargin   = 1.3cm,
                  xrightmargin  = 1.3cm,
                ]
    mySqrt := procedure(a) {
        if (a <= 1) {
            b := 1; 
        } else {
            b := a; 
        }     
        for (n in [1 .. 9]) {
            b := 1/2 * (b + a/b);
            print("$n$: $nDecimalPlaces(b, 100)$");
        }
        return b;
    };
\end{Verbatim}
\vspace*{-0.3cm}
  \caption{Ein \textsl{SetlX}-Programm zur iterativen Berechnung der Quadrat-Wurzel.}
  \label{fig:sqrt.stlx}
\end{figure} %\$

\begin{figure}[!ht]
  \centering
{\footnotesize
\begin{Verbatim}[ frame         = lines, 
                  framesep      = 0.3cm, 
                  labelposition = bottomline,
                  numbers       = none,
                  numbersep     = -0.2cm,
                  xleftmargin   = -0.5cm,
                  xrightmargin  = 0.5cm,
                ]
1: 1.5000000000000000000000000000000000000000000000000000000000000000000000000000000000000000000000000000
2: 1.4166666666666666666666666666666666666666666666666666666666666666666666666666666666666666666666666666
3: 1.4142156862745098039215686274509803921568627450980392156862745098039215686274509803921568627450980392
4: 1.4142135623746899106262955788901349101165596221157440445849050192000543718353892683589900431576443402
5: 1.4142135623730950488016896235025302436149819257761974284982894986231958242289236217849418367358303565
6: 1.4142135623730950488016887242096980785696718753772340015610131331132652556303399785317871612507104752
7: 1.4142135623730950488016887242096980785696718753769480731766797379907324784621070388503875343276416016
8: 1.4142135623730950488016887242096980785696718753769480731766797379907324784621070388503875343276415727
9: 1.4142135623730950488016887242096980785696718753769480731766797379907324784621070388503875343276415727
\end{Verbatim}
}
\vspace*{-0.3cm}
  \caption{Berechnung der Quadrat-Wurzel mit Hilfe der Folge $b_{n+1} = \bruch{1}{2}\cdot(b_n + \bruch{a}{b_n})$.}
  \label{fig:sqrt-output}
\end{figure}

\exercise
\begin{enumerate}
\item[(a)] Es seien $a,b \in \mathbb{R}$.  Die Folge $(a_n)_{n\in\mathbb{N}}$ werde durch Induktion wie folgt
           definiert:
           \begin{enumerate}
           \item $a_1 := a$,
           \item $a_2 := b$,
           \item $a_{n+2} := \frac{1}{2} \cdot (a_n + a_{n+1})$.
           \end{enumerate}
           Zeigen Sie, dass die Folge $(a_n)_{n\in\mathbb{N}}$ konvergiert und berechnen Sie den Grenzwert
           $\lim\limits_{n\rightarrow\infty} a_n$  in Abh�ngigkeit von den Startwerten $a$ und $b$.

           \noindent
           \textbf{Hinweis}: Die Gleichung f�r $a_{n+2}$ ist eine Rekurrenz-Gleichung, die Sie �ber den
           Ansatz $\alpha_n = \lambda^n$ l�sen k�nnen.  Sie werden dabei f�r $\lambda$ zwei m�gliche Werte
           $\lambda_1$ und $\lambda_2$ finden, die Sie f�r die L�sung in der Form
           \\[0.2cm]
           \hspace*{1.3cm}
           $a_n = c_1 \cdot \lambda_1^n + c_2 \cdot \lambda_2^n$
           \\[0.2cm]
           linear kombinieren m�ssen.  Die Koeffizienten $c_1$ und $c_2$ k�nnen Sie aus den Gleichungen f�r
           $a_1$ und $a_2$ bestimmen.  
\item[(b)] Es seien $a,b \in \mathbb{R}$ und zus�tzlich gelte $a > 0$ und $b > 0$.  
           Die Folge $(a_n)_{n\in\mathbb{N}}$ werde durch Induktion wie folgt definiert:
           \begin{enumerate}
           \item $a_1 := a$,
           \item $a_2 := b$,
           \item $a_{n+2} := \sqrt{a_n \cdot a_{n+1}}$.
           \end{enumerate}
           Zeigen Sie, dass die Folge $(a_n)_{n\in\mathbb{N}}$ konvergiert und berechnen Sie den Grenzwert
           $\lim\limits_{n\rightarrow\infty} a_n$ in  Abh�ngigkeit von den Startwerten $a$ und $b$. \eox

           \noindent
           \textbf{Bemerkung}: F�r zwei positive Zahlen $a$ und $b$ wird die Zahl $c$ f�r die 
           \\[0.2cm]
           \hspace*{1.3cm}
           $c := \sqrt{a \cdot b\;}$
           \\[0.2cm]
           gilt, als das \href{http://de.wikipedia.org/wiki/Geometrisches_Mittel}{\emph{geometrische Mittel}} von $a$ und $b$ bezeichnet.
\item[(b)] Es seien $a,b \in \mathbb{R}$ und zus�tzlich gelte $a > 0$ und $b > 0$.  
           Die Folge $(a_n)_{n\in\mathbb{N}}$ werde durch Induktion wie folgt definiert:
           \begin{enumerate}
           \item $a_1 := a$,
           \item $a_2 := b$,
           \item $\bruch{1}{a_{n+2}} := \bruch{1}{2} \cdot \left(\bruch{1}{a_n} + \bruch{1}{a_{n+1}}\right)$.
           \end{enumerate}
           Zeigen Sie, dass die Folge $(a_n)_{n\in\mathbb{N}}$ konvergiert und berechnen Sie den Grenzwert
           $\lim\limits_{n\rightarrow\infty} a_n$ in  Abh�ngigkeit von den Startwerten $a$ und $b$. \eox

           \noindent
           \textbf{Bemerkung}: F�r zwei positive Zahlen $a$ und $b$ wird die Zahl $c$ f�r die 
           \\[0.2cm]
           \hspace*{1.3cm}
           $\bruch{1}{c} := \bruch{1}{2} \cdot \left(\bruch{1}{a} + \bruch{1}{b}\right)$
           \\[0.2cm]
           gilt, als das \href{http://de.wikipedia.org/wiki/Harmonisches_Mittel}{\emph{harmonische Mittel}} von $a$ und $b$ bezeichnet.
\end{enumerate}



\section{Reihen}
\begin{Definition}[Reihe]
Ist $\folge{a_n}$ eine Folge, so definieren wir die Folge der \emph{Partial-Summen} $\folge{s_n}$
durch die Festsetzung
\\[0.2cm]
\hspace*{1.3cm}
$ s_n := \sum\limits_{i=1}^n a_i. $
\\[0.2cm]
Diese Folge bezeichnen wir auch als unendliche  \emph{Reihe}.  Die Folge $\folge{a_n}$
bezeichnen wir als die 
der Reihe $\folge{\sum_{i=0}^n a_i}$ \emph{zugrunde liegende Folge}.
Falls die Folge der Partial-Summen konvergiert, so schreiben wir den Grenzwert als
\\[0.2cm]
\hspace*{1.3cm}
$ \sum\limits_{i=1}^\infty a_i := \lim\limits_{n\rightarrow\infty} \sum\limits_{i=1}^n a_i. $
\eod
\end{Definition}

\noindent
Gelegentlich treten in der Praxis Folgen $\folge{a_n}$ auf, f�r welche die Folgenglieder
$a_i$ erst ab einem Index $k>1$ definiert sind.  
Um auch aus solchen Folge bequem Reihen bilden zu k�nnen, definieren wir in einem solchen
Fall die Partial-Summen $s_n$ durch
\\[0.2cm]
\hspace*{1.3cm}
$ s_n = \sum\limits_{i=k}^n a_i, $
\\[0.2cm] 
wobei wir vereinbaren, dass $\sum_{i=k}^n a_i = 0$ ist, falls $k > n$ ist.
\vspace*{0.3cm}

\begin{Satz}[Bernoullische Ungleichung]
Es sei $x \in \mathbb{R}$, $n \in \mathbb{N}_0$ und es gelte $x \geq -1$.   Dann gilt 
\\[0.2cm]
\hspace*{1.3cm}
$(1 + x)^n \geq 1 + n \cdot x$.
\\[0.2cm]
Diese Ungleichung wird als \emph{Bernoullische Ungleichung}
(\href{http://en.wikipedia.org/wiki/Jacob_Bernoulli}{Jakob Bernoulli}, 1655-1705)
bezeichnet.
\end{Satz}

\proof
Wir beweisen die Ungleichung durch vollst�ndige Induktion f�r alle $n \in \mathbb{N}_0$.
\begin{enumerate}
\item[I.A.:] $n = 0$.  Es gilt
             \\[0.2cm]
             \hspace*{1.3cm}
             $(1 + x)^0 = 1 \geq 1 = 1 + 0 \cdot x$. \quad  $\checkmark$
\item[I.S.:] $n \mapsto n + 1$.   Nach Induktions-Voraussetzung gilt
             \begin{equation}
               \label{eq:Bernoulli}
             (1 + x)^n \geq 1 + n \cdot x
             \end{equation}
             Da $x \geq -1$ ist, folgt $1 + x \geq 0$, so dass wir die Ungleichung
             \ref{eq:Bernoulli} mit $1 + x$ multiplizieren k�nnen.  Dann erhalten wir die 
             folgende Ungleichungs-Kette
             \\[0.2cm]
             \hspace*{1.3cm}
             $
             \begin{array}[t]{lcl}
             (1 + x)^{n+1} & \geq & (1 + n \cdot x) \cdot (1 + x)     \\[0.2cm]
                           & =    & 1 + (n + 1) \cdot x + n \cdot x^2 \\[0.2cm]
                           & \geq & 1 + (n + 1) \cdot x 
             \end{array}
             $
             \\[0.2cm]
             Also haben wir insgesamt
             \\[0.2cm]
             \hspace*{1.3cm}
             $(1 + x)^{n+1} \geq 1 + (n + 1) \cdot x$ 
             \\[0.2cm]
             gezeigt und das ist die Behauptung f�r $n+1$. $\checkmark$ \qed
\end{enumerate}

\begin{Satz}
Es sei $q \in \mathbb{R}$ mit $|q| < 1$.  Dann gilt
\\[0.2cm]
\hspace*{1.3cm}
$\lim\limits_{n\rightarrow\infty} q^n = 0$.
\end{Satz}

\proof
Wir nehmen zun�chst an, dass $q$ positiv ist. Aus $q < 1$ folgt dann 
\\[0.2cm]
\hspace*{1.3cm}
$1 < \bruch{1}{q}$ \quad und damit \quad $0 < \bruch{1}{q} - 1$. 
\\[0.2cm]
Wir definieren nun
\\[0.2cm]
\hspace*{1.3cm}
$x := \bruch{1}{q} - 1$. 
\\[0.2cm]
Mit Hilfe der Bernoullischen Ungleichung sehen wir nun, dass Folgendes gilt:
\\[0.2cm]
\hspace*{1.3cm}
$
\begin{array}[t]{lcl}
  \bruch{1}{q^n} & =    & \left(1 + \left(\bruch{1}{q} - 1\right)\right)^n  \\[0.4cm]
                 & \geq & 1 + n \cdot \left(\bruch{1}{q} - 1\right)         \\[0.4cm]
                 & =    & 1 + n \cdot x.
\end{array}
$
\\[0.2cm]
Durch Invertierung dieser Ungleichung erhalten wir
\\[0.2cm]
\hspace*{1.3cm}
$q^n \leq \bruch{1}{1 + n \cdot x}$
\\[0.2cm]
Ist nun ein $\varepsilon > 0$ gegeben, so definieren wir 
\\[0.2cm]
\hspace*{1.3cm}
$K := \left(\bruch{1}{\varepsilon} - 1\right) \cdot \bruch{1}{x} + 1$.
\\[0.2cm]
Dann gilt f�r alle $n \geq K$:
\\[0.2cm]
\hspace*{1.3cm}
$
\begin{array}[t]{lrcl}
            & \left(\bruch{1}{\varepsilon} - 1\right) \cdot \bruch{1}{x} + 1 & \leq & n             \\[0.4cm]
\Rightarrow & \left(\bruch{1}{\varepsilon} - 1\right) \cdot \bruch{1}{x} & < & n             \\[0.4cm]
\Rightarrow & \left(\bruch{1}{\varepsilon} - 1\right)                    & < & n \cdot x     \\[0.4cm]
\Rightarrow & \bruch{1}{\varepsilon}                                     & < & 1 + n \cdot x \\[0.4cm]
\Rightarrow & \bruch{1}{1 + n \cdot x} & < & \varepsilon
\end{array}
$
\\[0.2cm]
Insgesamt haben wir nun f�r alle $n \geq K$ gezeigt, dass
\\[0.2cm]
\hspace*{1.3cm}
$0 < q^n \leq \bruch{1}{1 + n \cdot x} < \varepsilon$
\\[0.2cm]
gilt, also haben wir f�r $n \geq K$
\\[0.2cm]
\hspace*{1.3cm}
$\left| q^n \right| < \varepsilon$.
\\[0.2cm]
F�r $q = 0$ ist diese Ungleichung offenbar auch g�ltig und wenn $q$ negativ ist, gilt $-q > 0$, so dass die
Ungleichung f�r $-q$ gilt:
\\[0.2cm]
\hspace*{1.3cm}
$\left| (-q)^n \right| < \varepsilon$.
\\[0.2cm]
Wegen $\left| (-q)^n \right| = \left| q^n \right|$ folgt daraus also, dass f�r alle $q$ die Ungleichung
\\[0.2cm]
\hspace*{1.3cm}
$\left| q^n \right| < \varepsilon$ \quad f�r $n \geq K$
\\[0.2cm]
g�ltig ist und damit ist die Behauptung bewiesen.  \qed

\noindent
Wir pr�sentieren nun einige Beispiele f�r konvergente Reihen:
\begin{enumerate}
\item Wir betrachten die Folge $\left(\bruch{1}{n\cdot (n+1)}\right)_{n\in\mathbb{N}}$.
      F�r die Partial-Summen zeigen wir durch Induktion �ber $n$, dass 
      \begin{equation}
        \label{eq:seq0}        
      \sum\limits_{i=1}^n \bruch{1}{i\cdot (i+1)} = 1 - \bruch{1}{n+1}
      \end{equation}
      gilt.
      \begin{enumerate}
      \item (Induktions-Anfang) $n=1$: Einerseits haben wir f�r $n=1$
           \\[0.2cm]
           \hspace*{1.3cm}      
           $\sum\limits_{i=1}^n \bruch{1}{i\cdot (i+1)} = \sum\limits_{i=1}^1 \bruch{1}{i\cdot (i+1)}
            = \bruch{1}{1\cdot (1+1)} = \bruch{1}{2}$,
           \\[0.2cm]
           andererseits gilt
           \\[0.2cm]
           \hspace*{1.3cm}      
           $1 - \bruch{1}{n+1} = 1 - \bruch{1}{1+1} = 1 - \bruch{1}{2} = \bruch{1}{2}$. $\checkmark$
      \item (Induktions-Schritt) $n \mapsto n+1$:  
            \\[0.2cm]
            \hspace*{1.3cm}      
            $
            \begin{array}{lcl}
              \sum\limits_{i=1}^{n+1} \bruch{1}{i\cdot (i+1)} 
              &               =  & \sum\limits_{i=1}^{n} \bruch{1}{i\cdot (i+1)} + \bruch{1}{(n+1)\cdot (n+2)} \\[0.3cm]
              & \stackrel{IV}{=} & 1 - \bruch{1}{(n+1)} + \bruch{1}{(n+1)\cdot (n+2)}                     \\[0.3cm]
              &               =  & 1 - \bruch{n+2 -1}{(n+1)\cdot (n+2)}                                  \\[0.3cm]
              &               =  & 1 - \bruch{n+1}{(n+1)\cdot (n+2)}                                     \\[0.3cm]
              &               =  & 1 - \bruch{1}{n+2}\quad \checkmark                                               
            \end{array}
            $
      \end{enumerate}
      Damit haben wir Gleichung (\ref{eq:seq0}) durch vollst�ndige Induktion nachgewiesen.
      Aus Gleichung (\ref{eq:seq0}) folgt nun 
      \\[0.2cm]
      \hspace*{1.3cm}      
      $\displaystyle\sum\limits_{i=1}^\infty \bruch{1}{i\cdot (i+1)} = \lim\limits_{n\rightarrow\infty} \sum\limits_{i=1}^n \bruch{1}{i\cdot (i+1)} = 
       \lim\limits_{n\rightarrow\infty} \Bigl(1 - \bruch{1}{n+1} \Bigr) = 1$.
\item Wir betrachten die Folge $\folge{q^n}$ f�r eine Zahl $q \in \mathbb{R}$.
      F�r die Partial-Summen gilt
      \\[0.2cm]
      \hspace*{1.3cm}      
      $s_n = \sum\limits_{i=0}^n q^i$.
      \\[0.2cm]
      Wir betrachten den Ausdruck $(1-q) \cdot  s_n$: 
      \\[0.2cm]
      \hspace*{1.3cm}      
      $
      \begin{array}{lcl}
       (1-q) \cdot  s_n & = & (1 - q) \cdot  \sum\limits_{i=0}^n q^i \\[0.3cm]
                   & = & \sum\limits_{i=0}^n q^i \;-\; q \cdot  \sum\limits_{i=0}^n q^i \\[0.3cm]
                   & = & \sum\limits_{i=0}^n q^i \;-\; \sum\limits_{i=0}^n q^{i+1} \\[0.3cm]
                   & = & \sum\limits_{i=0}^n q^i \;-\; \sum\limits_{i=1}^{n+1} q^{i} \\[0.3cm]
                   & = & \left(q^0 + \sum\limits_{i=1}^n q^i\right) \;-\; \left(\sum\limits_{i=1}^{n} q^{i} + q^{n+1}\right) \\[0.3cm]
                   & = & q^0 - q^{n+1} \\[0.3cm]
                   & = & 1 - q^{n+1} 
      \end{array}
      $
      \\[0.2cm]
      Es gilt also 
      \\[0.2cm]
      \hspace*{1.3cm}      
      $(1-q) \cdot  \sum\limits_{i=0}^{n} q^i = 1 - q^{n+1}$
      \\[0.2cm]
      Dividieren wir diese Gleichung durch $(1-q)$, so erhalten wir f�r die Partial-Summen
      den Ausdruck
      \\[0.2cm]
      \hspace*{1.3cm}      
      $\displaystyle \sum\limits_{i=0}^{n} q^i = \bruch{1 - q^{n+1}}{1-q}$.
      \\[0.2cm]
      Falls $|q| < 1$ ist, konvergiert die Folge $\folge{q^n}$ gegen $0$.  Damit gilt
      f�r $|q| < 1$
      \\[0.2cm]
      \hspace*{1.3cm} $\sum\limits_{i=0}^{\infty} q^i = \bruch{1}{1-q}$.
      \\[0.2cm]
      Die Reihe $\left(\sum\limits_{i=0}^{n} q^i\right)_{n\in\mathbb{N}}$ wird als
      \emph{geometrische Reihe} bezeichnet. \eox
\end{enumerate}
\pagebreak

\begin{Definition}[Alternierende Reihe] 
Hat eine Reihe die Form 
\\[0.2cm]
\hspace*{1.3cm}
$ \Folge{\sum\limits_{i=1}^{n}  (-1)^i \cdot  a_i} $
\\[0.2cm]
und gilt entweder
\\[0.2cm]
\hspace*{1.3cm}
$ \forall i \in \mathbb{N}: a_i \geq 0 \quad \mbox{oder} \quad \forall i \in \mathbb{N}: a_i \leq 0 $
\\[0.2cm]
so sprechen wir von einer \emph{alternierenden Reihe}.  \eod
\end{Definition}

\noindent
\textbf{Beispiel}: Die Reihe
\\[0.2cm]
\hspace*{1.3cm}
$\Folge{\sum\limits_{i=1}^{n} \bruch{(-1)^i}{i}}$
\\[0.2cm]
ist eine alternierende Reihe.  Wir werden sp�ter sehen, dass diese Reihe gegen den Wert $-\ln(2)$ konvergiert.
\eox


\begin{Definition}[Null-Folge]
Die Folge $\folge{a_n}$ ist eine \emph{Null-Folge} wenn gilt:
\\[0.2cm]
\hspace*{1.3cm}
$ \lim\limits_{n\rightarrow\infty} a_n = 0. $ \eod
\end{Definition}

\begin{Satz}[Leibniz-Kriterium, (\href{http://de.wikipedia.org/wiki/Leibniz}{Gottfried Wilhelm Leibniz}, 1646-1716)] \lb
Wenn die Folge $\folge{a_n}$ eine monoton fallende Null-Folge ist, dann 
konvergiert die alternierende Reihe
\\[0.2cm]
\hspace*{1.3cm}
$ \left(\sum\limits_{i=1}^{n} (-1)^i \cdot  a_i\right)_{n\in\mathbb{N}.} $
\end{Satz}
  
\noindent
\textbf{Beweis}:  Die  Partial-Summen $s_n$ sind durch
\\[0.2cm]
\hspace*{1.3cm}
$ s_n = \sum\limits_{i=1}^{n} (-1)^i \cdot  a_i $
\\[0.2cm]
definiert.  Wir betrachten zun�chst die Folge der Partial-Summen mit geraden Indizes, wir
betrachten also die Folge $\folge{s_{2 \cdot n}}$ und zeigen, dass diese Folge monoton fallend
ist.  Es gilt
\begin{equation}
  \label{eq:seq1}
 s_{2\cdot(n+1)} = s_{2\cdot n} + (-1)^{2\cdot n+1}\cdot a_{2 \cdot n+1} + (-1)^{2 \cdot n+2} \cdot  a_{2\cdot n+2} 
                 = s_{2 \cdot n} - a_{2 \cdot n+1} + a_{2 \cdot n+2}. 
\end{equation}
Daraus folgt
\\[0.2cm]
\hspace*{1.3cm}
$
\begin{array}{lrcl}
                 &   s_{2\cdot(n+1)} & \leq & s_{2\cdot n}       \\[0.2cm]
 \Leftrightarrow & s_{2\cdot n} - a_{2\cdot n+1} + a_{2\cdot n+2} & \leq &s_{2\cdot n}   \\[0.2cm]
 \Leftrightarrow &  a_{2\cdot n+2} & \leq & a_{2\cdot n+1}        
\end{array}
$
\\[0.2cm]
Die letzte Ungleichung ist aber nicht anderes als die Monotonie der Folge $\folge{a_n}$.

Als n�chstes zeigen wir durch vollst�ndige Induktion, dass die Folge der Partial-Summen
mit geraden Indizes nach unten beschr�nkt ist, genauer gilt
\\[0.2cm]
\hspace*{1.3cm}
$  s_{2\cdot n} \geq - a_1. $
\\[0.2cm]
Um dies nachzuweisen, zeigen wir durch vollst�ndige Induktion, dass f�r alle
$n\in\mathbb{N}_0$ gilt:
      \\[0.2cm]
      \hspace*{1.3cm}      
      $s_{2\cdot n+1} \geq - a_1$.
\begin{enumerate}
\item[I.A.:] $n=0$.
      \\[0.2cm]
      \hspace*{1.3cm} $s_{2\cdot 0+1} = s_1 = -a_1 \geq -a_1$.
\item[I.S.:] $n \mapsto n+1$ 
             \\[0.2cm]
\hspace*{1.3cm}
$ 
             \begin{array}{lcll}
               s_{2\cdot(n+1)+1} & = &  s_{2\cdot n+1} + a_{2\cdot n+2} - a_{2\cdot n+3}      \\[0.2cm]
                            & \geq & - a_1 + a_{2\cdot n+2} - a_{2\cdot n+3} & \mbox{nach Induktions-Voraussetzung}   \\[0.2cm]
                            & \geq & - a_1 & \mbox{wegen $a_{2\cdot n+2} \geq a_{2\cdot n+3}$.} \\[0.2cm]
                          
             \end{array}
             $
\\[0.2cm]
\end{enumerate}
Nun gilt f�r $n \in \mathbb{N}$
      \\[0.2cm]
      \hspace*{1.3cm}      
      $s_{2 \cdot n} = s_{2 \cdot n-1} + a_{2 \cdot n} \geq s_{2 \cdot n-1} \geq - a_1$.
      \\[0.2cm]
Da wir nun gezeigt haben, dass die Folge $\folge{s_{2 \cdot n}}$ sowohl monoton fallend als auch
nach unten beschr�nkt ist, folgt aus Satz \ref{satz:monoton}, dass diese Folge konvergent ist.
Der Grenzwert dieser Folge sei $s$:
\\[0.2cm]
\hspace*{1.3cm}
$ s := \lim\limits_{n\rightarrow\infty} s_{2 \cdot n}. $
\\[0.2cm]
Dann konvergiert auch die Folge $\folge{s_{n}}$ gegen $s$. Dies sehen wir wie folgt:
Sei $\varepsilon > 0$ gegeben.  Weil   $\folge{s_{2 \cdot n}}$ gegen $s$ konvergiert gibt es eine
Zahl
$K_1$, so dass f�r alle $n \geq K_1$ die Ungleichung
\begin{equation}
  \label{eq:seq2}
  \bigl| s_{2 \cdot n} - s \bigr| < \bruch{1}{2}\cdot\varepsilon
\end{equation}
erf�llt ist.  Weil $\folge{a_{n}}$ eine Null-Folge ist gibt es au�erdem eine Zahl $K_2$,
so dass f�r alle $n \geq K_2$ die Ungleichung
\begin{equation}
  \label{eq:seq3}
  \bigl| a_n - 0 \bigr| < \bruch{1}{2} \cdot \varepsilon
\end{equation}
gilt.  Wir setzen $K:= \max(2 \cdot K_1 + 1,K_2)$ und zeigen, dass f�r alle $n \geq K$ die Ungleichung
\\[0.2cm]
\hspace*{1.3cm}
$ \bigl| s_n - s \bigr| < \varepsilon $
\\[0.2cm]
gilt.  Wir erbringen diesen Nachweis �ber eine Fall-Unterscheidung:
\begin{enumerate}
\item $n$ ist gerade, also gilt $n= 2 \cdot m$.
      \\[0.2cm]
\hspace*{1.3cm}
$ \begin{array}{lcl}
        \bigl| s_n - s \bigr| & = & \bigl| s_{2 \cdot m} - s \bigr| \\[0.2cm]
                              & < & \bruch{1}{2}\cdot\varepsilon \\[0.2cm]
                              & < & \varepsilon,
         \end{array}
      $
\\[0.2cm]
      denn aus $n=2 \cdot m$ und $n \geq K$ folgt $m \geq K_1$.
\item $n$ ist ungerade, also gilt $n= 2 \cdot m+1$.
      \\[0.2cm]
\hspace*{1.3cm}
$ \begin{array}{lcl}
        \bigl| s_n - s \bigr| & = & \bigl| s_{2 \cdot m+1} - s \bigr| \\[0.2cm]
                              & = & \bigl| s_{2 \cdot m+1} - s_{2 \cdot m} + s_{2 \cdot m} - s \bigr| \\[0.2cm]
                              & \leq & \bigl| s_{2 \cdot m+1} - s_{2 \cdot m} \bigr| + \bigl| s_{2 \cdot m} - s \bigr| \\[0.2cm]
                              & < & \bigl| a_{2 \cdot m+1} \bigr| + \bruch{1}{2}\cdot\varepsilon \\[0.2cm]
                              & < & \bruch{1}{2}\cdot\varepsilon + \bruch{1}{2}\cdot\varepsilon \\[0.2cm]
                              & = & \varepsilon,
         \end{array}
       $
\\[0.2cm]
       denn aus $n=2\cdot m+1$ und $n \geq K$ folgt $m \geq K_1$ und $n \geq K_2$.
\end{enumerate}
Damit ist der Beweis abgeschlossen. \hspace*{\fill} $\Box$ 
\vspace*{0.3cm}
\pagebreak

\begin{Satz}[Cauchy'sches Konvergenz-Kriterium f�r Reihen] \lb
Die Reihe $\folge{\sum_{i=0}^{n} a_i}$ ist genau dann konvergent, wenn es f�r alle 
$\varepsilon>0$ eine Zahl $K$ gibt, so dass gilt:
\\[0.2cm]
\hspace*{1.3cm}
$ \forall n,l \in \mathbb{N}: n \geq K \rightarrow \left|\sum\limits_{i=n+1}^{n+l} a_i \right| < \varepsilon.$
\\[0.2cm]
  
\end{Satz}

\noindent
\textbf{Beweis}:  Nach den S�tzen aus dem Abschnitt �ber Folgen ist die Folge
$\folge{s_n}$ der durch
\\[0.2cm]
\hspace*{1.3cm}
$ s_n = \sum\limits_{i=0}^{n} a_i $
\\[0.2cm]
definierten Partial-Summen genau dann konvergent, wenn $\folge{s_n}$ eine Cauchy-Folge
ist, wenn also gilt:
\\[0.2cm]
\hspace*{1.3cm}
$ \forall \varepsilon \in \mathbb{R}_+: \exists K \in \mathbb{R}: \forall m,n \in \mathbb{N}: 
   m \geq K \wedge n \geq K \rightarrow \bigl| s_m - s_n \bigr| < \varepsilon.
$
\\[0.2cm]
In der letzen Formel k�nnen wir ohne Einschr�nkung der Allgemeinheit annehmen,
dass $n \leq m$ gilt.  Dann ist $m = n + l$ f�r eine nat�rliche Zahl $l$.  Setzen wir hier
die Definition der Partial-Summen ein, so erhalten wir
\\[0.2cm]
\hspace*{1.3cm}
$ \begin{array}{lcl}
   \left| s_{m} - s_n \right| & = & \left| s_{n+l} - s_n \right| \\[0.2cm]
                              & = & \left| \sum\limits_{i=1}^{n+l} a_i - \sum\limits_{i=1}^n a_i \right| \\[0.4cm]
                              & = & \left| \sum\limits_{i=n+1}^{n+l} a_i \right| \\[0.2cm]
   \end{array}
$
\\[0.2cm]
und damit ist klar, dass die Ungleichung des Satzes �quivalent dazu ist,
dass die Folge der Partial-Summen eine Cauchy-Folge ist.
\hspace*{\fill} $\Box$

\begin{Korollar} \lb
Wenn die Reihe $\Reihe{a_n}$ konvergent ist, dann ist die Folge $\folge{a_n}$ eine Null-Folge.  
\end{Korollar}

\noindent
\textbf{Beweis}: Nach dem Cauchy'schen Konvergenz-Kriterium gilt \\[0.2cm]
\hspace*{1.3cm}      
$\forall \varepsilon \in \mathbb{R}_+:\exists K \in \mathbb{R}:\forall n,l \in \mathbb{N}: n \geq K \rightarrow \left|\sum\limits_{i=n+1}^{n+l} a_i \right| < \varepsilon$.
\\[0.2cm]
Setzen wir hier $l=1$ so haben wir insbesondere
\\[0.2cm]
\hspace*{1.3cm}
$\forall \varepsilon \in \mathbb{R}_+:\exists K \in \mathbb{R}:\forall n \in \mathbb{N}: n \geq K \rightarrow \left|\sum\limits_{i=n+1}^{n+1} a_i \right| < \varepsilon$. 
\\[0.2cm]
Wegen \\[0.2cm]
      \hspace*{1.3cm}      
      $\left|\sum\limits_{i=n+1}^{n+1} a_i \right| = |a_{n+1}|$
      \\[0.2cm]
folgt also 
\\[0.2cm]
\hspace*{1.3cm}
$\forall \varepsilon \in \mathbb{R}_+:\exists K \in \mathbb{R}:\forall n \in \mathbb{N}: n \geq K \rightarrow |a_{n+1}| < \varepsilon$.      
\\[0.2cm]
Diese Formel dr�ckt aus, dass $\folge{a_n}$ eine Null-Folge ist.
\hspace*{\fill} $\Box$ 
\vspace*{0.3cm}

\noindent
Mit Hilfe des letzten Satzes k�nnen wir zeigen, dass die \emph{harmonische Reihe}
\\[0.2cm]
\hspace*{1.3cm}
$ \Folge{\sum\limits_{i=1}^{n} \bruch{1}{i}} $
\\[0.2cm]
divergiert.  W�re diese Reihe konvergent, so g�be es nach dem Cauchy'schen
Konvergenz-Kriterium eine Zahl $K$, so dass f�r alle $n \geq K$ und alle $l$ die Ungleichung
\\[0.2cm]
\hspace*{1.3cm}
$ \sum\limits_{i=n+1}^{n+l} \bruch{1}{i} < \bruch{1}{2} $
\\[0.2cm]
gilt.  Insbesondere w�rde diese Ungleichung dann f�r $l=n$ gelten. 
F�r beliebige $n$ gilt aber die folgende Absch�tzung:
\\[0.2cm]
\hspace*{1.3cm}
$
  \sum\limits_{i=n+1}^{n+n} \bruch{1}{i}  \geq  \sum\limits_{i=n+1}^{2 \cdot n} \bruch{1}{2 \cdot n} 
 = n \cdot  \bruch{1}{2 \cdot n} = \bruch{1}{2} 
$
\\[0.2cm]
Damit erf�llt die harmonische Reihe das Cauchy'sche Konvergenz-Kriterium nicht.

\begin{Satz}[Majoranten-Kriterium]
F�r die Folgen $\folge{a_n}$  und  $\folge{b_n}$  gelte:
\begin{enumerate}
\item $\forall n \in \mathbb{N}: 0 \leq a_n \leq b_n$.
\item Der Grenzwert $\sum\limits_{i=1}^\infty b_i$ existiert.
\end{enumerate}
Dann existiert auch der Grenzwert $\sum\limits_{i=1}^\infty a_i$.
\end{Satz}


\noindent
\textbf{Beweis}:  Es gilt
\\[0.2cm]
\hspace*{1.3cm}
$ \sum\limits_{i=1}^n a_i \leq \sum\limits_{i=1}^n b_i \leq \sum\limits_{i=1}^\infty b_i. $
\\[0.2cm]
Also ist die Folge $\Folge{\sum\limits_{i=1}^{n} a_i}$ monoton wachsend und beschr�nkt und
damit konvergent.
\vspace*{0.3cm}

\remark
Oft wird im Majoranten-Kriterium die Voraussetzung
\\[0.2cm]
\hspace*{1.3cm}
$\forall n \in \mathbb{N}: 0 \leq a_n \leq b_n$ 
\\[0.2cm] 
abgeschw�cht zu 
\\[0.2cm]
\hspace*{1.3cm}      
$\forall n \in \mathbb{N}: n \geq K \rightarrow 0 \leq a_n \leq b_n$.
\\[0.2cm]
Hierbei ist $K$ dann eine geeignet gew�hlte Schranke.  Die G�ltigkeit dieser Form des
Majoranten-Kriteriums folgt aus der Tatsache, dass das Ab�ndern endlich vieler
Glieder einer Reihe f�r die Frage, ob eine Reihe konvergent ist, unbedeutend ist.

\exercise
Zeigen Sie mit dem Majoranten-Kriterium, dass die Reihe 
$\Folge{\sum\limits_{i=1}^{n} \bruch{1}{i^2}}$ konvergiert.
\vspace*{0.3cm}


\noindent
\textbf{L�sung}:  Es gilt
\\[0.2cm]
\hspace*{1.3cm}
$
\begin{array}{lcl}
            & i + 1\geq i & \mid\; \cdot  (i+1) \\[0.2cm]
\Rightarrow & (i+1)^2 \geq i\cdot (i + 1) & \mid \bruch{1}{\cdot} \\[0.2cm]
\Rightarrow & \bruch{1}{(i+1)^2} \leq \bruch{1}{i\cdot (i + 1)} 
\end{array}
$
\\[0.2cm]
Damit ist die Reihe $\Folge{\sum\limits_{i=1}^{n}\bruch{1}{i\cdot (i+1)}}$
eine konvergente Majorante der Reihe $\Folge{\sum\limits_{i=1}^{n}\bruch{1}{(i+1)^2}}$.  Wegen
\\[0.2cm]
\hspace*{1.3cm}
$ \sum\limits_{i=1}^{\infty} \bruch{1}{i^2} = \bruch{1}{1^2} + \sum\limits_{i=1}^{\infty} \bruch{1}{(i+1)^2} $
\\[0.2cm]
folgt die Konvergenz aus dem Majoranten-Kriterium.
\vspace*{0.3cm}

\noindent
\textbf{Bemerkung}: Wir werden sp�ter zeigen, dass 
\\[0.2cm]
\hspace*{1.3cm}
$ \sum\limits_{i=1}^{\infty} \bruch{1}{i^2} = \bruch{\;\pi^2}{6\;} \quad\mbox{gilt.} $



\begin{Satz}[Minoranten-Kriterium]
F�r die Folgen $\folge{a_n}$  und  $\folge{b_n}$  gelte:
\begin{enumerate}
\item $\forall n \in \mathbb{N}: 0 \leq a_n \leq b_n$.
\item Der Grenzwert $\sum\limits_{i=1}^\infty a_i$ existiert nicht.
\end{enumerate}
Dann existiert auch der Grenzwert $\sum\limits_{i=1}^\infty b_i$ nicht.
\end{Satz}

\noindent
\textbf{Beweis}:  Wir f�hren den Beweis indirekt und nehmen an, dass 
$\sum\limits_{i=1}^\infty b_i$
existiert.  Nach dem Majoranten-Kriterium m��te dann auch der Grenzwert
$\sum\limits_{i=1}^\infty a_i$
existieren und dass steht im Widerspruch zur Voraussetzung. \hspace*{\fill} $\Box$
\vspace*{0.3cm}

\exercise
Zeigen Sie, dass die Reihe 
\\[0.2cm]
\hspace*{1.3cm} $\Reihe{\bruch{1}{\sqrt{i\,}}}$ \\[0.2cm]
nicht konvergiert.
\vspace*{0.3cm}

\noindent
\textbf{L�sung}:  Wir benutzen das Minoranten-Kriterium und zeigen, dass die
Reihe $\Reihe{\bruch{1}{n}}$ eine divergente Minorante der Reihe
  $\Reihe{\bruch{1}{\sqrt{n}}}$ ist:
\\[0.2cm]
\hspace*{1.3cm}
$
\begin{array}{llcll}
                 & \bruch{1}{n}   & \leq & \bruch{1}{\sqrt{n}}  & \mid \; \bruch{1}{\cdot}\\[0.3cm]
\Leftrightarrow  & n              & \geq & \sqrt{n}             & \mid \;\cdot^2  \\[0.3cm]
\Leftrightarrow  & n^2            & \geq & n                    & \mid \; \cdot \bruch{1}{n}  \\[0.3cm]
\Leftrightarrow  & n              & \geq & 1                    & \mid \; 
\end{array}
$
\\[0.2cm]
Da die letzte Ungleichung offenbar f�r alle $n\in \mathbb{N}$ wahr ist, ist der Beweis abgeschlossen.
\hspace*{\fill} $\Box$


\begin{Satz}[Quotienten-Kriterium]
Es sei $\folge{a_n}$ eine Folge und $q\in\mathbb{R}$ eine Zahl, so dass gilt:
\begin{enumerate}
\item $0 \leq q < 1$
\item $\forall n \in \mathbb{N}: 0 \leq a_n$
\item $\forall n \in \mathbb{N}: a_{n+1} \leq q \cdot  a_n$
\end{enumerate}
Dann konvergiert die Reihe $\Reihe{a_i}$.
\end{Satz}

\noindent 
\textbf{Beweis}:  Wir zeigen, dass die geometrische Reihe $\Reihe{a_0\cdot q^i}$ eine konvergente Majorante der
Reihe $\Reihe{a_n}$ ist.  Dazu zeigen wir durch Induktion �ber $n$ dass folgendes gilt:
\\[0.2cm]
\hspace*{1.3cm}
$\forall n \in \mathbb{N}: a_{n} \leq a_1 \cdot q^{n-1}$
\begin{enumerate}
\item[I.A.]: $n = 1$.  Wegen $q^0 = 1$ gilt trivialerweise
      \\[0.2cm]
      \hspace*{1.3cm} $a_1  \leq a_1 \cdot  q^0$.
\item[I.S.]: $n \mapsto n+1$.  Es gilt: 
      \\[0.2cm]
      \hspace*{1.3cm}      
      $
      \begin{array}{lclll}
        a_{n+1} & \leq & q \cdot  a_n        & \mbox{nach Voraussetzung} \\[0.2cm]
                & \leq & q \cdot  a_1 \cdot  q^{n-1}  & \mbox{nach Induktions-Voraussetzung} \\[0.2cm]
                & =    & a_1 \cdot  q^{n}. & &\hspace*{2cm} \Box
              \end{array}
      $
\end{enumerate}
\vspace*{0.3cm}

\noindent
\textbf{Bemerkung}:  Beim Quotienten-Kriterium sind eigentlich nur die Betr�ge der
Folgenglieder $|a_n|$ wichtig, denn es l��t sich folgende Versch�rfung des
Quotienten-Kriteriums zeigen:  
Ist $\folge{a_n}$ eine Folge, $q\in\mathbb{R}$ und $K \in \mathbb{R}$, so dass 
\\[0.2cm]
\hspace*{1.3cm}
$ 0 \leq q < 1 \quad \wedge \quad \forall n \in \mathbb{N}: n \geq K \rightarrow \left|\bruch{a_{n+1}}{a_n}\right| \leq q  $
\\[0.2cm]
gilt.  Dann konvergiert die Reihe $\Reihe{a_n}$.

\noindent
\textbf{Beispiel}:
Wir zeigen mit Hilfe des Quotienten-Kriteriums, dass die Reihe
$\Reihe{\bruch{z^i}{i!}}$ f�r alle $z \in \mathbb{C}$ konvergiert.  F�r $z=0$ ist die Konvergenz der
Reihe trivial und sonst betrachten wir
den Quotienten $a_{n+1}/a_n$ f�r diese Reihe, setzen $K = 2\cdot |z|$ und $q = \bruch{1}{2}$
und zeigen, dass das Quotienten-Kriterium erf�llt ist, denn f�r alle $n\geq K$ gilt:
\\[0.2cm]
\hspace*{1.3cm}      
$\left|\bruch{a_{n+1}}{a_n}\right| =  \bruch{\;\bruch{|z^{n+1}|}{\;(n+1)!\;}\;}{\bruch{\;|z^{n}|\;}{n!}} 
                            =  \bruch{\;|z^{n+1}| \cdot  n!\;}{\;|z^{n}| \cdot  (n+1)!\;} 
                            =  \bruch{\;|z|\;}{\;n+1\;} 
                            \leq  \bruch{\;|z|\;}{\;K\;} 
                            = \bruch{\;|z|\;}{\;2\cdot |z|\;} 
                            =  \bruch{\;1\;}{\;2\;} 
$.

\begin{Satz}[Wurzel-Kriterium]
Es sei $\folge{a_n}$ eine Folge und $q\in\mathbb{R}$ eine Zahl, so dass 
\begin{enumerate}
\item $0 \leq q < 1$
\item $\forall n \in \mathbb{N}: 0 \leq a_n$
\item $\forall n \in \mathbb{N}: n > 0 \rightarrow \sqrt[n]{a_n} \leq q$
\end{enumerate}
gilt.  Dann konvergiert die Reihe $\Reihe{a_n}$.
\end{Satz}

\noindent 
\textbf{Beweis}: Auch hier k�nnen wir den Nachweis der Konvergenz dadurch f�hren indem wir
zeigen, dass die geometrische Reihe $\Reihe{q^i}$ eine konvergente Majorante ist: F�r
$n>0$ gilt
\\[0.2cm]
\hspace*{1.3cm} $a_n \leq q^n \;\Leftrightarrow\; \sqrt[n]{a_n} \leq q$.
\hspace*{\fill} $\Box$
\vspace*{0.3cm}

\noindent
\textbf{Bemerkung}:  Beim Wurzel-Kriterium sind eigentlich nur die Betr�ge der
Folgenglieder $|a_n|$ wichtig, denn es l��t sich folgende Versch�rfung des
Quotienten-Kriteriums zeigen:  
Ist $\folge{a_n}$ eine Folge, $q\in\mathbb{R}$ und $K \in \mathbb{R}$, so dass 
\\[0.2cm]
\hspace*{1.3cm}
$ 0 \leq q < 1 \quad \wedge \quad \forall n \in \mathbb{N}: n \geq K \rightarrow \sqrt[n]{|a_n|} \leq q  $
\\[0.2cm]
gilt.  Dann konvergiert die Reihe $\Reihe{a_n}$.


\noindent
\textbf{Beispiel}: Wir zeigen mit dem Wurzel-Kriterium, dass die Reihe
$\Reihe{\bruch{1}{i!}}$konvergiert.  Wir setzen
$K = 4$ und $q = \bruch{1}{2}$.  Zun�chst k�nnen Sie mit vollst�ndiger Induktion
leicht zeigen, dass f�r alle nat�rlichen Zahlen $n\geq 4$ die Ungleichung
$n! \geq 2^n$ gilt.  Damit haben wir f�r $n\geq 4$:
\\[0.2cm]
\hspace*{1.3cm}
$ \sqrt[n]{\bruch{1}{n!}} \leq \bruch{1}{2}     \quad \Leftrightarrow \quad
   \bruch{1}{n!} \leq \left(\bruch{1}{2}\right)^n  \quad \Leftrightarrow \quad
   n! \geq 2^n. 
$



\section{Potenz-Reihen}
Es bezeichne $x$ eine Variable und $\folge{a_n}$ sei eine Folge von Zahlen.  Dann
bezeichnen wir den Ausdruck
      \\[0.2cm]
      \hspace*{1.3cm} $\sum\limits_{n=0}^\infty a_n \cdot  x^n$      
      \\[0.2cm]
als \emph{formale Potenz-Reihe}.  Wichtig ist hier, dass $x$ keine feste Zahl ist, sondern
eine Variable, f�r die wir sp�ter reelle (oder auch komplexe) Zahlen einsetzen.  Wenn wir
in einer Potenz-Reihe f�r $x$ eine feste Zahl einsetzen, wird aus der Potenz-Reihe eine
gew�hnliche Reihe.  Der Begriff der Potenz-Reihen kann als eine Verallgemeinerung des
Begriffs des   Polynoms aufgefa�t werden. 
\vspace*{0.3cm}

\noindent
\textbf{Beispiele}:
\begin{enumerate}
\item $\sum\limits_{n=0}^\infty \bruch{x^n}{n!}$ ist eine formale Potenz-Reihe.  
      Wir haben oben mit Hilfe des Quotienten-Kriteriums gezeigt, dass diese Reihe f�r
      beliebige reelle Zahlen konvergiert.
\item $\sum\limits_{n=1}^\infty \bruch{x^n}{n}$ ist eine formale Potenz-Reihe.  
      Setzen wir f�r $x$ den Wert $1$ ein, so erhalten wir die divergente harmonische
      Reihe.  F�r $x=-1$ erhalten wir eine alternierende Reihe, die nach dem Leibniz-Kriterium
      konvergent ist.
\end{enumerate}
In der Theorie der Potenz-Reihen ist die Frage entscheidend, welche Zahlen wir f�r die
Variable $x$ einsetzen k�nnen, so dass die Reihe konvergiert.  Diese Frage wird durch die
folgenden S�tze beantwortet.

\begin{Satz}\label{satz:konvergenz-radius}
Wenn die Potenz-Reihe 
\\[0.2cm]
\hspace*{1.3cm}
$\sum\limits_{n=0}^\infty a_n\cdot x^n$ 
\\[0.2cm]
f�r einen Wert $u \in \mathbb{C}$ konvergiert, dann konvergiert die Reihe auch f�r alle $v
\in \mathbb{C}$, f�r die $|v| < |u|$ ist.
\end{Satz}

\noindent 
\textbf{Beweis}:  Da die Reihe $\sum_{n=0}^\infty a_n\cdot u^n$ konvergiert, folgt aus dem
Korollar zum Cauchy'schen Konvergenz-Kriterium, dass die Folge $\folge{a_n\cdot u^n}$ eine
Null-Folge ist.  Also gibt es eine Zahl $K$, so dass f�r alle $n \geq K$ die Ungleichung 
\\[0.2cm]
\hspace*{1.3cm}      
$\left|a_n\cdot u^n\right| \leq 1$
\\[0.2cm]
gilt.  Wir definieren 
\\[0.2cm]
\hspace*{1.3cm}
$q := \left|\bruch{v}{u}\right|$. 
\\[0.2cm]
 Aus $|v| < |u|$ folgt $q < 1$.
Dann haben wir f�r alle $n \geq K$ die folgende Absch�tzung:
      \\[0.2cm]
      \hspace*{1.3cm}      
      $\left|a_n\cdot v^n\right| = 
      \left|a_n\cdot u^n\right| \cdot  \left|\bruch{v^n}{u^n}\right| =
      \left|a_n\cdot u^n\right| \cdot  q^n \leq q^n$.
      \\[0.2cm]
Diese Absch�tzung zeigt, dass die geometrische Reihe eine konvergente Majorante der Reihe
$a_n\cdot v^n$ ist.  Damit folgt die Konvergenz der Reihe $\sum_{n=0}^\infty a_n\cdot v^n$  aus dem
Majoranten-Kriterium. \hspace*{\fill} $\Box$ 

\begin{Satz}
Wenn die Potenz-Reihe $\sum_{n=0}^\infty a_n\cdot x^n$ f�r einen Wert $u \in \mathbb{C}$
divergiert, dann divergiert die Reihe auch f�r alle $v \in \mathbb{C}$, f�r die
$|u| < |v|$ ist.
\end{Satz}

\noindent
\textbf{Beweis}:  W�rde die Reihe $\sum_{n=0}^\infty a_n\cdot v^n$ konvergieren,
dann m��te nach Satz \ref{satz:konvergenz-radius} auch die Reihe 
$\sum_{n=0}^\infty a_n\cdot u^n$ konvergieren. \hspace*{\fill} $\Box$
\vspace*{0.3cm}

\noindent
Die letzten beiden S�tze erm�glichen es nun, den Begriff \emph{Konvergenz-Radius} zu
definieren.  Es sei 
\\[0.2cm]
\hspace*{1.3cm}
$\sum\limits_{n=0}^\infty a_n\cdot x^n$ 
\\[0.2cm]
eine formale Potenz-Reihe.
Wenn diese Reihe f�r alle $x\in\mathbb{C}$ konvergiert, dann sagen wir, dass der
Konvergenz-Radius den Wert $\infty$ hat.  Andernfalls definieren wir den Konvergenz-Radius
als
\\[0.2cm]
\hspace*{1.3cm}      
$R := \sup \left\{ |u| \;\Big|\; u \in \mathbb{C} \wedge
                   \sum\limits_{n=0}^\infty a_n\cdot u^n \;\mathrm{konvergiert} 
         \right\}
$.
\\[0.2cm]
Aus den letzten beiden S�tzen folgt dann:
\begin{enumerate}
\item $\forall z \in \mathbb{C}: |z| < R \rightarrow \sum\limits_{n=0}^\infty a_n\cdot z^n$ konvergiert. 
\item $\forall z \in \mathbb{C}: |z| > R \rightarrow \sum\limits_{n=0}^\infty a_n\cdot z^n$ divergiert. 
\end{enumerate}
In der Gau�'schen Zahlen-Ebene ist die Menge 
$\bigl\{ z \in \mathbb{C} \;\big|\; |z| < R \bigr\}$ das Innere eines Kreises mit dem
Radius $R$ um den Nullpunkt.
Der folgende Satz gibt uns eine effektive M�glichkeit, den Konvergenz-Radius zu berechnen.

\begin{Satz}
Es sei 
\\[0.2cm]
\hspace*{1.3cm}
$\sum\limits_{n=0}^\infty a_n\cdot z^n$ 
\\[0.2cm]
eine formale Potenz-Reihe und die Folge
\\[0.2cm]
\hspace*{1.3cm}
$\Folge{\bruch{|a_n|}{|a_{n+1}|}}$
\\[0.2cm]
 konvergiere.   Dann ist der Konvergenz-Radius $R$ durch
folgende Formel gegeben:
      \\[0.2cm]
      \hspace*{1.3cm}      
      $R = \lim\limits_{n\rightarrow\infty} \left|\bruch{a_n}{\;a_{n+1}\;}\right|$.  
\end{Satz}

\noindent
\textbf{Beweis}:
Es sei $u \in \mathbb{C}$ mit $|u| < R$.  In diesem Fall m�ssen wir zeigen, dass die Reihe
\\[0.2cm]
\hspace*{1.3cm}
$\sum\limits_{n=0}^\infty a_n\cdot u^n$
\\[0.2cm]
 konvergiert.  Wir werden diesen Nachweis mit Hilfe
Quotienten-Kriteriums erbringen.  Wir setzen $q := \bruch{|u|}{R} < 1$ und zeigen, dass f�r
alle hinreichend gro�en $n$ die Ungleichung 
      \\[0.2cm]
      \hspace*{1.3cm}      
      $\bruch{|a_{n+1}\cdot u^{n+1}|}{|a_{n}\cdot u^{n}|} \leq q$
      \\[0.2cm]
erf�llt ist.  Aus 
      \\[0.2cm]
      \hspace*{1.3cm}      
       $R = \lim\limits_{n\rightarrow\infty} \left|\bruch{a_n}{\;a_{n+1}\;}\right|$
      \\[0.2cm]
folgt, dass es f�r beliebige $\varepsilon >0$ eine Zahl $K$ gibt, so dass f�r alle $n \geq K$
die Ungleichung
      \\[0.2cm]
      \hspace*{1.3cm}      
      $\left| \Bigl|\bruch{a_{n}}{a_{n+1}}\Bigr| - R \right| < \varepsilon$
      \\[0.2cm]
gilt.  Wir setzen $\varepsilon := \bruch{1}{2}(R - |u|)$. Wir zeigen, dass dann f�r alle 
$n \geq K$ die Ungleichung $ \left|\bruch{a_{n}}{a_{n+1}}\right| > \bruch{1}{2}(R + |u|)$ gilt:
\\[0.2cm]
\hspace*{1.3cm}
$
\begin{array}{cl}
 & \left|\;\bigl|\bruch{a_{n}}{a_{n+1}}\bigr| + \bigl(R - \bigl|\bruch{a_{n}}{a_{n+1}}\bigr|\bigr)\;\right| = |R| = R  \\[0.5cm]
\Rightarrow     & \bigl|\bruch{a_{n}}{a_{n+1}}\bigr| + \bigl|\; \bigl(R - \bigl|\bruch{a_{n}}{a_{n+1}}\bigr|\bigr)\;\bigr| \geq R  \\[0.5cm]
\Rightarrow     & \bigl|\bruch{a_{n}}{a_{n+1}}\bigr| + \varepsilon > R  \\[0.5cm]
\Rightarrow     & \bigl|\bruch{a_{n}}{a_{n+1}}\bigr| + \bruch{1}{2}(R - |u|) > R  \\[0.5cm]
\Rightarrow     & \bigl|\bruch{a_{n}}{a_{n+1}}\bigr|   > R - \bruch{1}{2}(R - |u|)  \\[0.5cm]
\Rightarrow     & \bigl|\bruch{a_{n}}{a_{n+1}}\bigr|   > \bruch{1}{2}(R + |u|).  \\[0.5cm]
\end{array}
$
\\[0.2cm]
Jetzt k�nnen wir zeigen, dass die Reihe $\sum_{n=0}^\infty a_n\cdot u^n$ das
Quotienten-Kriterium erf�llt, denn f�r alle $n \geq K$ gilt:
\\[0.2cm]
\hspace*{1.3cm}
$ \left|\bruch{a_{n+1}\cdot u^{n+1}}{a_n\cdot u^n}\right| = 
   \left|\bruch{a_{n+1}}{a_n}\right| \cdot  |u| = 
   \bruch{|u|}{\left|\bruch{a_{n}}{a_{n+1}}\right|} < 
   \bruch{|u|}{\bruch{1}{2}(R + |u|)} < 
   \bruch{|u|}{\bruch{1}{2}(R + R)} =
   \bruch{|u|}{R} = q
$
\\[0.2cm]
Um den Beweis abzuschlie�en m�ssen wir noch zeigen, die Reihe $\sum_{n=0}^\infty a_n\cdot u^n$
divergiert wenn  $R < |u|$ ist.   Dies folgt aus der Tatsache, dass die Folge
$\folge{a_n\cdot u^n}$ f�r $|u|>R$ keine Null-Folge ist.  Die Details bleiben dem Leser
�berlassen. 
\hspace*{\fill} $\Box$ 
\vspace*{0.3cm}

\noindent
\textbf{Bemerkung}:
Der obige Satz bleibt auch richtig, wenn
      \\[0.2cm]
      \hspace*{1.3cm}      
      $\lim\limits_{n\rightarrow\infty} \left|\bruch{a_n}{\;a_{n+1}\;}\right| = \infty$
      \\[0.2cm]
ist, denn dann ist die Potenz-Reihe $\sum_{n=0}^\infty a_n\cdot x^n$ f�r alle $u\in\mathbb{C}$
konvergent.
\vspace*{0.3cm}

\noindent
\textbf{Beispiel}:
Die Potenz-Reihe $\sum_{n=1}^\infty \bruch{x^n}{n}$ hat den Konvergenz-Radius $R=1$, denn
es gilt 
\\[0.2cm]
\hspace*{1.3cm}
$\lim\limits_{n\rightarrow\infty} \left|\frac{\frac{1}{n}}{\;\frac{1}{n+1}\;}\right| = 
       \lim\limits_{n\rightarrow\infty} \bruch{n+1}{n} = 
       1 + \lim\limits_{n\rightarrow\infty}  \bruch{1}{n} = 1 + 0 = 1.
$



\begin{Satz}[Hadamard]
Es sei $\sum_{n=0}^\infty a_n\cdot x^n$ eine Potenz-Reihe und die Folge
$\Folge{\sqrt[n]{|a_n|}}$ konvergiere.   Dann gilt 
      \\[0.2cm]
      \hspace*{1.3cm}      
      $\displaystyle\bruch{1}{R} = \lim\limits_{n\rightarrow\infty} \sqrt[n]{|a_n|}$.  
\end{Satz}

\noindent
\textbf{Bemerkung}:
Setzen wir $\bruch{1}{\infty} = 0$, so  bleibt die Formel 
      \\[0.2cm]
      \hspace*{1.3cm}      
      $\bruch{1}{R} = \lim\limits_{n\rightarrow\infty} \sqrt[n]{|a_n|}$.  
      \\[0.2cm]
auch in dem Fall
$\lim\limits_{n\rightarrow\infty} \sqrt[n]{|a_n|} = 0$ richtig, denn dann gilt $R = \infty$.

\vspace*{0.3cm}

\noindent
\textbf{Beispiel}:
Die Potenz-Reihe $\sum_{n=1}^\infty \bruch{x^n}{n^n}$ hat den Konvergenz-Radius $R=\infty$, denn
es gilt 
\\[0.2cm]
\hspace*{1.3cm}
$ \lim\limits_{n\rightarrow\infty} \sqrt[n]{\bruch{1}{n^n}} =  
   \lim\limits_{n\rightarrow\infty} \bruch{1}{n} = 0.
$



%%% Local Variables: 
%%% mode: latex
%%% TeX-master: "analysis"
%%% End: 

\include{stetige-funktionen}
\include{anwendungen}
\include{integration}
\include{irrational}
\include{fourier}
\include{rundungsfehler}

%\bibliographystyle{alpha}
\bibliography{/Users/karldrstroetmann/Dropbox/Kurse/cs}
%\bibliography{/Users/stroetma/Dropbox/Kurse/cs}

\end{document}



